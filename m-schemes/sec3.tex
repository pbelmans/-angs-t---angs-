\section{Affine schemes}
The anti-equivalence from the previous \iftex section\fi\ifblog post\fi
\begin{equation}
  \text{affine schemes} \underset{\text{equivalence}}{\overset{\text{anti-}}{\longleftrightarrow}} \text{commutative and unital rings}.
  \label{equation:schemes-anti-equivalence-2}
\end{equation}
basically says it all, but it will require an explanation in more detail to really grasp what an \emph{affine scheme} is. Remark that I've been talking about affine varieties and affine schemes for some time now, in the next \iftex section\fi\ifblog post\fi I will elaborate on this, but they are the crucial building blocks for general schemes.

There are three parts in the definition of an affine scheme: it is a \emph{set}, equipped with a certain \emph{topology} and there is a \emph{structure sheaf} defined on this topological space that captures its geometric properties. The first two parts are quite familiar, but for most of you (speaking to my fellow students) the latter will be new.

\iftex
\subsection[The set Spec A]{The set $\Spec A$}
\fi
\ifblog
\subsection{The set $\Spec A$}
\fi
Let~$A$ be a commutative and unital ring. Define the \emph{set~$\Spec A$} to be the set of all prime ideals of~$A$. We already see a difference with the variety case: we were only interested in the maximal spectrum there. Let's look at some examples to see what this implies.

\begin{example}
  \begin{equation}
    \Spec\mathbb{Z}=\left\{ \left[ (p) \right]\,|\,\text{$p\in\mathbb{Z}$ prime} \right\}\cup\left\{ [(0)] \right\}.
  \end{equation}
\end{example}

\begin{example}
  \begin{equation}
    \Spec\mathbb{C}[x]=\left\{ \left[ (x-\alpha) \right]\,|\,\alpha\in\mathbb{C} \right\}\cup\left\{ [(0)] \right\}.
  \end{equation}
\end{example}

\begin{example}
  \begin{equation}
    \Spec\mathbb{C}[x,y]=\left\{ \text{irreducible polynomials in~$\mathbb{C}[x,y]$} \right\}\cup\left\{ [(0)] \right\}.
  \end{equation}
\end{example}

The first two examples are pretty familiar:~$\Spec\mathbb{C}[x]$ is the affine line as we know it from our varieties course (plus an extra point~$[(0)]$) and while~$\mathbb{Z}$ doesn't really look like a reduced and finitely generated~$\mathbb{C}$-algebra we already know from previous lectures in the seminar its spectrum looks like~$\Spec\mathbb{C}[x]$. We observe that an extra point came to life here too.

But it's really the third examples that will show the difference between the variety-approach and the scheme-approach. It might be looking familiar (it's just~$\mathbb{A}^2_{\mathbb{C}}$ from last year, right?) but it is radically different. \emph{Every} irreducible polynomial is now a point! We were used to points corresponding to maximal ideals~$[(x-\alpha,y-\beta)]$ and from our previous examples we knew about the presence of~$[(0)]$ but stuff like~$[(y^2-x^3-x)]$ have become points too. Our new points correspond to irreducible curves in the affine plane.

For reasons that will become clear once we've introduced the topology on the set~$\Spec A$ we call our familiar points (i.e., corresponding to maximal ideals) \emph{closed points} while all the others (including~$[(0)]$, if~$A$ is a domain but not a field) are \emph{generic points}. Remark that the generic point corresponding to a curve does not contain the maximal points that lie on it in the usual sense, but we will observe something that approximates this inclusion when we have our topology.

\iftex
\subsection[Zariski topology on Spec A]{Zariski topology on $\Spec A$}
\fi
\ifblog
\subsection{Zariski topology on $\Spec A$}
\fi
As in the case we with varieties we call the topology that an (affine) scheme is equipped with the \emph{Zariski topology} as it is defined in a similar vein. We will define what a closed set is (unlike most topologies that are defined in terms of the open sets, but there's no difference) is and then we check whether finite unions and arbitrary intersections are again closed.

By a closed subset of the set~$\Spec A$ we mean a set
\begin{equation}
  \vanishing(I)=\left\{ [\mathfrak{p}]\in\Spec A\,|\,I\subseteq\mathfrak{p} \right\}
\end{equation}
where~$I$ is an ideal of~$A$. Remark that this is completely analogous to the varieties we considered in our course last year, but we'll now put emphasis on certain consequences of this particular definition.

If we consider the examples from the previous section we obtain the following descriptions of their topologies. By the analogy between~$\Spec\mathbb{Z}$ and~$\mathbb{C}[x]$ we have in both cases a cofinite topology. The closed subsets are:
\begin{itemize}
  \item the entire space (corresponding to~$I=(0)$);
  \item the empty set (now we take~$I=A$);
  \item finite unions of closed points~$\left\{ [\mathfrak{p}_1],\ldots,[\mathfrak{p}_k] \right\}$ with~$\mathfrak{p}_i$ a prime ideal generated by either a prime number in~$\mathbb{Z}$ of a linear term in~$\mathbb{C}[x]$ (hence we use~$I=\mathfrak{p}_1\cdots\mathfrak{p}_k$ for the definition of a closed set).
\end{itemize}

In the case where~$A=\mathbb{C}[x,y]$ it will become obvious why we speak about closed and generic points. If we take~$I=\mathfrak{m}=(x-\alpha,y-\beta)$ we see that what we called a closed point in the previous section is actually closed in the Zariski topology! On the other hand, if we take~$I=(f(x,y))$ which corresponds to a generic point we see that the closed set~$\vanishing(I)$ contains the generic point (closures don't remove points) and all the closed points that are somehow contained in the curve that's described by the generic point. Other closed sets are finite unions of basic closed sets, the empty set and the entire space.

Proving that this is indeed a topology (i.e., taking finite unions and arbitrary intersections is legal) is left as an exercise to the reader. You might consider the topology on~$\Spec\mathbb{C}[x,y]$ first, before tackling the~$\Spec A$ case but it's really easy and a straightforward generalization from last year.

Hence we can stop talking about ``the set~$\Spec A$'' and instead start using ``the topological space~$\Spec A$'', when we equip the set with the Zariski topology.


\subsection{The structure sheaf}
So far everything we did was rather familiar, translating varieties to schemes by replacing reduced and finitely generated~$\mathbb{C}$-algebras by commutative and unital rings. But in order to get a real anti-equivalence in~\eqref{equation:schemes-anti-equivalence-2} and a fully operational scheme theory we need to consider the topological space~$\Spec A$ together with its structure sheaf. This idea isn't new though: it boils down to a \emph{vast} generalization of the coordinate ring of an affine variety.

We wish to consider an element~$f\in A$ as a \emph{function} on~$\Spec A$, just like we had regular functions on an affine variety which were elements of the corresponding coordinate ring. But we won't restrict ourselves to regular functions (which are defined everywhere) but we'll describe what the regular functions on all open subsets of the topological space~$\Spec A$ look like! This may seem artificial now, but it will turn out to be crucial in the development of algebraic geometry in terms of schemes.

This specific instance of ``regular functions on a scheme'' is called the \emph{structure sheaf}, but to understand it, I'll define sheaves in general. Let's first repeat what an abelian category is.

\begin{definition}
  A category is said to be \emph{abelian} if it has zero objects, pullbacks and pushouts and all mono- and epimorphisms are normal.
\end{definition}

The exact meaning of this is not important (it just says that it is a nice category, for a certain notion of nice) but a small set of examples will be useful. It will suffice if this set of examples contains abelian groups and (left-) modules over a (not necessarily commutative) ring. Remark that the first is a special case of the latter.

\begin{definition}
  A \emph{presheaf}~$\mathcal{F}$ on a topological space~$(X,\mathcal{T})$ over an abelian category~$\mathcal{K}$ is given by
  \begin{enumerate}
    \item an object~$\mathcal{F}(U)\in\mathcal{K}$ for every open set~$U\subseteq X$ which are the \emph{sections} over~$U$;
    \item for every inclusion~$V\subseteq U$ a \emph{restriction morphism}~$\res^U_V\colon\mathcal{F}(U)\to\mathcal{F}(V)$ which is a morphism in~$\mathcal{K}$.
  \end{enumerate}

  These restriction morphisms satisfy the following conditions/
  \begin{enumerate}
    \item $\res^U_U=\identity_{\mathcal{F}(U)}$;
    \item if we have~$W\subseteq V\subseteq U$ then~$\res^U_W=\res^V_W\circ\res^U_V$.
  \end{enumerate}
\end{definition}

Now to get to the definition of a sheaf we'll introduce two axioms that a presheaf must satisfy: the local identity and the gluing axiom.

\begin{definition}
  A presheaf~$\mathcal{F}$ is a \emph{sheaf} if and only if
  \begin{description}
    \item[\textbf{local identity}] if $U=\bigcup_{i\in I}U_i$ is an open covering with~$s$ and~$t$ in~$\mathcal{F}(U)$ such that they are equal in all the restrictions then~$s=t$;
    \item[\textbf{gluing}] if we are given~$s_i\in\mathcal{F}(U_i)$ in an open covering~$U=\bigcup_{i\in I}U_i$ such that all restrictions to intersections~$U_i\cap U_j$ agree then we can construct a section~$s\in\mathcal{F}(U)$ that restricts to each of the~$s_i$.
  \end{description}
\end{definition}

\begin{example}
  The \emph{constant presheaf} that assigns a fixed object~$A\in\mathcal{K}$ to every open set is the easiest example imaginable. But this is not a sheaf (can you see why?), unless we \emph{sheafify} it which corresponds to taking ``the best sheaf we can make out of a given presheaf''. If we do this we get a copy of~$A$ in~$\mathcal{F}(X)$ for every connected component of~$X$.
\end{example}

\begin{example}
  Another standard example is the \emph{skyscraper sheaf}. Let's fix an element~$x\in X$ and assign~$A\in\mathcal{K}$ to every~$U$ that contains~$x$ but assign~$0$ otherwise. Can you see why this is called a skyscraper?
\end{example}

\begin{example}
  And the guiding examples of actual sheaves are sheaves of continuous functions on a topological space to another topological space, or ($k$ times continuously) differentiable functions on a manifold (with codomain~$\mathbb{R}$). The restriction morphisms are easily seen to be the restriction of functions, and as continuity or differentiability is a local property it is compatible with gluing.
  
  Remark that continuous functions in general do not admit any abelian category structure (as the codomain of the continuous functions needs to admit an algebraic structure in an abelian category!), so we have a sheaf of sets in this case. That's legal too, but we'll be interested in sheaves over abelian categories most of the time. Interesting fact: all the sheaves over a topological space with values in an abelian category form an abelian category.
\end{example}

Now if we take our topological space to be~$\Spec A$ we can define its \emph{structure sheaf}~$\mathcal{O}_{\Spec A}$ which will capture all of the structure we'll need in algebraic geometry. Remember from our varieties course that we have a basis of distinguished open sets, where a distinguished open is defined by
\begin{equation}
  \distinguished(f)=\left\{ [\mathfrak{p}]\in\Spec A\,|\,f\notin\mathfrak{p} \right\}
\end{equation}
such that~$f$ is a function on~$\Spec A$, or an element of~$A$. It corresponds to a region where~$f$ does not vanish. In order to understand this, consider~$\Spec\mathbb{Z}$ and look at what happens to~$n\in\mathbb{Z}$ when you take the quotient~$\mathbb{Z}/\mathfrak{p}$ at each point~$[\mathfrak{p}]$ of~$\Spec\mathbb{Z}$.

It can be shown that it suffices to define the structure sheaf on these distinguished open sets. So let's do it:
\begin{equation}
  \mathcal{O}_{\Spec A}(\distinguished(f))=A_f
\end{equation}
where~$A_f$ denotes the localization of~$A$ at the element~$f$.

That's it. We've defined what it means to be an affine scheme. It corresponds to taking a commutative and unital ring, turning it into a set with a certain topology and then equip it with a structure sheaf. The anti-equivalence in~\eqref{equation:schemes-anti-equivalence-2} is now given by taking~$\Spec$ in one direction and considering the ring of global sections which is just~$\mathcal{O}_{\Spec A}({\Spec A})$, all the functions that are defined on the entire space.
