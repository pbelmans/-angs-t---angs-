\section{Affine schemes}
The anti-equivalence from the previous \iftex section\fi\ifblog post\fi
\begin{equation}
  \text{affine schemes} \underset{\text{equivalence}}{\overset{\text{anti-}}{\longleftrightarrow}} \text{commutative and unital rings}.
\end{equation}
basically says it all, but it will require an explanation in more detail to really grasp what an \emph{affine scheme} is. Remark that I've been talking about affine varieties and affine schemes for some time now, in the next \iftex section\fi\ifblog post\fi I will elaborate on this, but they are the crucial building blocks for general schemes.

There are three parts in the definition of an affine scheme: it is a \emph{set}, equipped with a certain \emph{topology} and there is a \emph{structure sheaf} defined on this topological space that captures its geometric properties. The first two parts are quite familiar, but for most of you (speaking to my fellow students) the latter will be new.

\subsection{The set $\Spec A$}

\subsection{Zariski topology on $\Spec A$}

\subsection{The structure sheaf}

