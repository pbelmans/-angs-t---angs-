\iftex
\section[What we want from a geometry over F\_1]{What we want from a geometry over $\mathbb{F}_1$}
\fi
\ifblog
\section{What we want from a geometry over $\mathbb{F}_1$}
\fi

From now on the main object of study is~$\mathbb{F}_1$. I will first elaborate on what Kapranov-Smirnov calls the ``folklore imagery'', trying to understand the motivation behind a statement like

\begin{quote}
  A vectorspace of~$\mathbb{F}_1$ is just a set.
\end{quote}

Then I will draw some conclusions from these observations. To conclude this post I'll give a short ``philosophical'' discussion of how we can approach a geometry over~$\mathbb{F}_1$, drawing analogies to noncommutative geometry and representation theory. This part has become slightly superfluous after the seminar of October 21 as some of this has been introduced back then, but I will give my take on it for completeness' sake.


\iftex
\subsection[Demystifying the F\_1-lore]{Demystifying the $\mathbb{F}_1$-lore}
\fi
\ifblog
\subsection{Demystifying the $\mathbb{F}_1$-lore}
\fi
