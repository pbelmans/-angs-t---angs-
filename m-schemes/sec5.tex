\iftex
\section{What we want from a geometry over \texorpdfstring{$\mathbb{F}_1$}{F\_1}}
\fi
\ifblog
\section{What we want from a geometry over $\mathbb{F}_1$}
\fi

From now on the main object of study is~$\mathbb{F}_1$. I will first elaborate on what Kapranov-Smirnov calls the ``folklore imagery'', trying to understand the motivation behind a statement like

\begin{quote}
  A vector space of~$\mathbb{F}_1$ is just a set.
\end{quote}

Afterwards I will draw some conclusions from these observations, motivating some of the approaches outlined in \iftex\cite{mapping-fun}\fi\ifblog\href{http://www.neverendingbooks.org/index.php/f_un-and-braid-groups.html}{Mapping $\mathbb{F}_1$-land}\fi.


\iftex
\subsection{Demystifying the \texorpdfstring{$\mathbb{F}_1$}{F\_1}-lore}
\fi
\ifblog
\subsection{Demystifying the $\mathbb{F}_1$-lore}
\fi
In \iftex \cite{projective-geometry-over-f1} \fi\ifblog \href{http://arxiv.org/abs/math/0407093}{Projective geometry over $\mathbb{F}_1$ and the Gaussian binomial coefficient} \fi a nice build-up is given. I will summarise and give my own viewpoint.

We already know that~$\mathbb{F}_1$ doesn't exist as a field (because we need~$0\neq 1$). But what if we say the trivial ring should be taken as~$\mathbb{F}_1$? In that case modules over this ring are in relationship with vector spaces over~$\mathbb{F}_1$. But there are only trivial modules over the trivial ring, hence no nontrivial vector spaces. This approach obviously doesn't work. A conclusion I like to make after vastly generalising this approach:

\begin{quote}
  Setting~$\mathbb{F}_1$ to be \emph{something} doesn't work.
\end{quote}

The idea of looking at vector spaces looks promising though: given a field it is the most obvious structure built upon it, so we want to make sense of it over $\mathbb{F}_1$. We know that the cardinality of an~$n$\nobreakdash-dimensional vector space~$V$ over a finite field~$\mathbb{F}_q$ is~$q^n$. Applying this to~$\mathbb{F}_1$ we get a single point. So far so good, but we also know that a \emph{basis} for~$V$ consists of~$n$ elements. The problem with this approach is that it's too direct: we cannot \emph{construct} objects over~$\mathbb{F}_1$, we need to get there by an analogy that avoids contradictions like this. Applying this for instance to~$\mathbb{F}_1[t]$ we see that this doesn't yield any satisfying definition either. The conclusion after another round of generalisation:

\begin{quote}
  Look at induced objects.
\end{quote}

The same applies to noncommutative geometry by the way. But let's focus on~$\mathbb{F}_1$ for the moment.

So we need a simple object over~$\mathbb{F}_1$ where we can avoid an explicit construction, getting facts about that object only by analogy without running into contradictions. Let's try this for~$\mathbb{P}^n/\mathbb{F}_1$. The construction of this over an actual field~$k$ consists of constructing~$\mathbb{A}^{n+1}/k$ (an~$n+1$\nobreakdash-dimensional vector space over~$k$) and setting~$\mathbb{P}^n/k$ to be the set of lines through the origin.

If we take~$k=\mathbb{F}_q$ the number of points in~$\mathbb{A}^{n+1}/\mathbb{F}_q$ is~$q^{n+1}$. The number of lines through the origin is~$(q^{n+1}-1)/(q-1)$: just take any point in~$\mathbb{A}^{n+1}/\mathbb{F}_q\setminus\left\{ 0 \right\}$, this defines a line through the origin, but there are~$q$ points (including the origin) on this line, therefore we divide by~$q-1$. If we write down the polynomial function counting the number of points in~$\mathbb{P}^n/\mathbb{F}_q$ (i.e.,~$q^n+q^{n-1}+\ldots+1$) and evaluate in~$q=1$ we see something that leads to a nontrivial object! 

\begin{quote}
  The projective geometry~$\mathbb{P}^1/\mathbb{F}_1$ contains~$n$ points.
\end{quote}

What we have actually done is changing the construction of~$\mathbb{P}^n/\mathbb{F}_1$ from an \emph{algebraic-geometric} viewpoint to a \emph{gubernatorial-geometric} viewpoint, something that is quintessential in finite geometry (and I guess I'm the angs+'er with the most background in this kind of stuff :)). If people are interested in a write-up about this, I'd be happy to provide one but I suspect my fellow seminarians are not that into finite geometry and combinatorics. The only important observation we need to make is that a line in~$\mathbb{P}^n/\mathbb{F}_1$ contains exactly two points in this sense, as a line in~$\mathbb{P}^n/\mathbb{F}_q$ contains~$q+1$ points by the axioms of a combinatorial-geometric projective space.

Now taking $\mathbb{F}_1$-vector spaces to be sets actually makes sense: $\mathbb{A}^{n+1}/\mathbb{F}_1$ is an~$(n+1)$-set, taking any point as the distinguished base point (or origin) and considering ``lines'' through the origin, or more appropriately $2$-subsets, of which we have exactly~$n$ as we started with~$n+1$ elements and fixed an origin. This notion of fixing an origin will have benefits later on, so keep this in mind.

\subsection{Implications}

Now we can switch our attention to \iftex\cite{kapranov-smirnov}\fi\ifblog Kapranov's and Smirnov's unfinished paper\fi. If Wittgenstein were an algebraic geometer interested in $\mathbb{F}_1$-geometry I guess he would have written a Tractatus Absoluto-Geometricus, which could have looked (in a \emph{very} crude sense) like this
\begin{enumerate}
  \item Geometry over~$\mathbb{F}_1$ can only be understood through induced objects.
    \item Vector spaces over~$\mathbb{F}_1$ are plain sets.
      \begin{enumerate}
        \item Dimension equals cardinality.
        \item $\mathrm{GL}_n(\mathbb{F}_1)=\mathrm{S}_n$.
        \item $\mathrm{SL}_n(\mathbb{F}_1)=\mathrm{A}_n$.
        \item $\det\colon\mathrm{GL}_n(\mathbb{F}_1)\to\mathbb{F}_1^\times$ is the sign homomorphism.
        \item The Grassmannian~$\mathrm{Gr}(k,n)(\mathbb{F}_1)$ is the set of~$k$-subsets.
      \end{enumerate}

    \item The polynomial ring $\mathbb{F}_1[t]$ can only be understood through its automorphisms.
      \begin{enumerate}
        \item Polynomial automorphisms are a generalisation of field automorphisms by evaluation at ``zero''.
        \item We have $\mathrm{GL}_n(\mathbb{F}_1[t])\to\mathrm{GL}_n(\mathbb{F}_1)$.
        \item $\mathrm{GL}_n(\mathbb{F}_1[t])=\mathrm{B}_n$ the braid group on~$n$ strings, by analogy of the canonical~$\mathrm{B}_n\to\mathrm{S}_n$.
        \item For more information I refer you to the grandmaster himself and his blog post \href{http://www.neverendingbooks.org/index.php/f_un-and-braid-groups.html}{$\mathbb{F}_1$ and braid groups}.
      \end{enumerate}

    \item Finite fields (and therefore $\mathbb{F}_1$) have finite extensions and algebraic closures.
      \begin{enumerate}
        \item $\mathbb{F}_{1^n}$ as a vector space over~$\mathbb{F}_1$ is the (pointed) set consisting of the $n$-th roots of unity~$\mu_n$ and the origin.
        \item $\mathbb{A}^1/\mathbb{F}_1$ as a scheme is~$\Spec\mathbb{F}_1[t]$.
        \item $\Spec\mathbb{F}_1[t]$ describes the algebraic closure of $\mathbb{F}_1$.
        \item $\overline{\mathbb{F}_1}$ therefore corresponds to $\mathrm{\mu}_\infty\cup\left\{ 0 \right\}$.
      \end{enumerate}

    \item We can generalise linear algebra to finite extensions.
      \begin{enumerate}
        \item The action of $\mathbb{F}_{1^n}$ on~$V$ is the action of $\mu_n$ on~$V\setminus\left\{ 0 \right\}$.
        \item $\mathbb{F}_{1^n}$-vector spaces are nothing but $\mu_n$-sets.
        \item This action is multiplicative, there is no additive structure.
        \item The lack of additive structure agrees with the notion of vector spaces as (pointed) sets.
        \item A $\mathbb{F}_{1^n}$-basis is a set containing a representative of every orbit under the $\mu_n$-action.
      \end{enumerate}
\end{enumerate}

For the real Wittgenstein aficionado, my apologies for not using his actual numbering system. This post mostly served as a way for me to get all my $\mathbb{F}_1$-folklore straight, so that I can explain to a complete outsider why stuff in the $\mathbb{F}_1$ is taken as what it is. If you feel any gaps present, please tell me so.

If you followed me this far you should be familiar enough with the ideas and twists of mind necessary to understanding parts of Kapranov-Smirnov, or you have made up your mind and will take all $\mathbb{F}_1$-stuff to be dadaist nonsense. But before tackling Kapranov-Smirnov I suggest you read the series of posts developing the same theme as this one did but in far greater depth:
\begin{enumerate}
  \item \href{http://www.neverendingbooks.org/index.php/the-f_un-folklore.html}{The $\mathbb{F}_1$ folklore}
  \item \href{http://www.neverendingbooks.org/index.php/absolute-linear-algebra.html}{Absolute linear algebra}
  \item \href{http://www.neverendingbooks.org/index.php/f_un-and-braid-groups.html}{$\mathbb{F}_1$ and braid groups}
\end{enumerate}

For the next (and last) post of this series, the main idea will be
\begin{quote}
  Stuff over $\mathbb{F}_1$ contains multiplicative structure, not additive structure.
\end{quote}

