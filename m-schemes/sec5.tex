\iftex
\section{What we want from a geometry over \texorpdfstring{$\mathbb{F}_1$}{F\_1}}
\fi
\ifblog
\section{What we want from a geometry over $\mathbb{F}_1$}
\fi

From now on the main object of study is~$\mathbb{F}_1$. I will first elaborate on what Kapranov-Smirnov calls the ``folklore imagery'', trying to understand the motivation behind a statement like

\begin{quote}
  A vectorspace of~$\mathbb{F}_1$ is just a set.
\end{quote}

Afterwards I will draw some conclusions from these observations, motivating some of the approaches outlined in \iftex\cite{mapping-fun}\fi\ifblog Mapping $\mathbb{F}_1$-land\fi.


\iftex
\subsection[Demystifying the F\_1-lore]{Demystifying the $\mathbb{F}_1$-lore}
\fi
\ifblog
\subsection{Demystifying the $\mathbb{F}_1$-lore}
\fi
In \iftex \cite{projective-geometry-over-f1} \fi\ifblog \href{http://arxiv.org/abs/math/0407093}{Projective geometry over $\mathbb{F}_1$ and the Gaussian binomial coefficient} \fi a nice build-up is given. I will summarize and give my own viewpoint.

We already know that~$\mathbb{F}_1$ doesn't exist as a field (because we need~$0\neq 1$). But what if we say the trivial ring should be taken as~$\mathbb{F}_1$. In that cases modules over this ring are in relation with vectorspaces over~$\mathbb{F}_1$. But there are only trivial modules over the trivial ring, hence no nontrivial vectorspaces. This approach doesn't work. The conclusion after generalizing:

\begin{quote}
  Setting~$\mathbb{F}_1$ to be \emph{something} doesn't work.
\end{quote}

The idea of looking at vectorspaces looks promising though: given a field it's the most obvious structure built upon it. We know that the cardinality of an~$n$\nobreakdash-dim\-ensional vectorspace~$V$ over a finite field~$\mathbb{F}_q$ is~$q^n$. Applying this to~$\mathbb{F}_1$ we get a single point. So far so good, but we also know that a \emph{basis} for~$V$ consists of~$n$ elements. The problem with this approach is that it's too direct: we cannot \emph{construct} objects over~$\mathbb{F}_1$, we need to get there by an analogy that avoids contradictions like this. Applying this for instance to~$\mathbb{F}_1[t]$ we see that this doesn't yield any satisfying definition either. The conclusion after generalizing:

So we need a simple object over~$\mathbb{F}_1$ where we can avoid an explicit construction, getting facts about that object purely by analogy without running into contradictions. Let's try this for~$\mathbb{P}^n/\mathbb{F}_1$. The construction of this over an actual field~$k$ consists of constructing~$\mathbb{A}^{n+1}/k$ (an~$n+1$\nobreakdash-dimensional vectorspace over~$k$) and setting~$\mathbb{P}^n/k$ to be the set of lines through the origin.

If we take~$k=\mathbb{F}_q$ the number of points in~$\mathbb{A}^{n+1}/\mathbb{F}_q$ is~$q^{n+1}$. The number of lines through the origin is~$(q^{n+1}-1)/(q-1)$: just take any point in~$\mathbb{A}^{n+1}/\mathbb{F}_q\setminus\left\{ 0 \right\}$, this defines a line through the origin, but there are~$q$ points (including the origin) on this line, therefore we divide by~$q-1$. If we write down the polynomial function counting the number of points in~$\mathbb{P}^n/\mathbb{F}_q$ (i.e.,~$q^n+q^{n-1}+\ldots+1$) and evaluate in~$q=1$ we see something that leads to a nontrivial object! 

\begin{quote}
  The projective geometry~$\mathbb{P}^1/\mathbb{F}_1$ contains~$n$ points.
\end{quote}

\subsection{Implications}
