\section{Schemes}

In my previous post I introduced affine schemes, which will serve as the building blocks for schemes. Besides constructing schemes in general I will also say a few words on projective schemes, which can serve as another class of building blocks for schemes.

The second part of this post will be devoted to \emph{properties} of schemes. I will introduce some of the more frequently occurring, so that we can understand them when they arise in \iftex\cite{mapping-fun}\fi\ifblog Mapping~$\mathbb{F}_1$-land\fi.

\subsection{Schemes and projective schemes}
I could (or should) have introduced this terminology in my previous post but I haven't. Let's patch up our knowledge in order to define schemes.

\begin{definition}
  When we equip a topological space~$X$ with a sheaf of rings~$\mathcal{O}_X$ (like we did!) we will call~$(X,\mathcal{O}_X)$ a \emph{ringed space}. So affine schemes are examples of ringed spaces, just like differentiable manifolds. These will be our two guiding examples.
\end{definition}

\begin{definition}
  The \emph{pushforward} of a sheaf~$\mathcal{F}$ on~$X$ through a continuous function~$f\colon X\to Y$ is given by~$f_\ast\mathcal{F}(V)=\mathcal{F}(f^{-1}(V))$ where~$V\subseteq Y$ open. Give it a moment's thought as to why this is again a sheaf.
\end{definition}

\begin{definition}
  Now an \emph{isomorphism of ringed spaces}~$(X,\mathcal{O}_X)$ and~$(Y,\mathcal{O}_Y)$ is given by a homeomorphism~$f\colon X\to Y$ and an isomorphism of sheaves ~$\mathcal{O}_Y$ and~$f_\ast\mathcal{O}_X$. The definition of an isomorphism of sheaves is left as an exercise. Summarizing: we have a bijection of the underlying sets which serves as an ``isomorphism'' in the topological sense and an isomorphism of sheaves. Now in my previous post I defined an affine scheme to be a tuple~$(\Spec A,\mathcal{O}_{\Spec A})$, from now on we will say that \emph{any ringed space that is isomorphic to such a tuple} is an \emph{affine scheme}.
\end{definition}

\begin{definition}
  The restriction of a sheaf~$\mathcal{O}_X$ on a topological space~$X$ to an open subspace~$U$ is defined by taking~$\mathcal{O}_{X|U}(V)=\mathcal{O}_X(V)$ for~$V\subseteq U$ open.
\end{definition}

We are now ready for the objective of this post.

\begin{definition}
  A \emph{scheme} $(X,\mathcal{O}_X)$ is a ringed space such that any point~$x\in X$ has an open neighbourhood~$U$ such that~$(U,\mathcal{O}_{X|U})$ is an affine scheme.
\end{definition}

This definition looks a lot like the definition of a (differentiable) manifold. In that case every affine scheme is taken to be an open subset of~$\mathbb{R}^n$, which gives us the opportunity of considering charts. Now the ``model'' of what the neighbourhood of a point looks like is more general, but the idea is analogous. Just like in manifolds this will have a great impact on how many proofs in algebraic geometry with schemes work: all properties that are in a sense affine (or local) can be proved for general schemes by reducing it to the affine case. It even borders on the annoying how some algebraic geometry text books seem to consist of reductions to the affine case.

It's time for some examples. These are taken from \iftex\cite{foag}\fi\ifblog \href{http://math216.wordpress.com}{Ravi Vakil's notes}\fi, more precisely Section~5.4, appropriately called Three examples. I will discuss two of these and give an exercise.

\begin{example}
  Take your favourite algebraically closed field~$k$ and let~$A=k[x,y]$. In that case we'll denote~$\Spec A$ by~$\mathbb{A}^2_k$, the affine plane over the field~$k$. Now consider the open subset~$U=\mathbb{A}^2_k\setminus\left\{ (0,0) \right\}$, or when we write~$(0,0)$ as the point corresponding the ideal that vanishes there:~$\mathbb{A}^2_k\setminus\left\{ \left[ (x,y) \right] \right\}$. This is \emph{not a distinguished set}! Remember that distinguished opens correspond to complements of closed sets cut out by a \emph{single} function. But in~$\mathbb{A}^2_k$ these correspond to curves, hence the complement of the origin is not a distinguished open set. It is the union of two distinguished open sets though:~$\mathrm{D}(x)$ and~$\mathrm{D}(y)$.

  The sheaf of algebraic functions gives for these two open sets the rings~$k[x,y,x^{-1}]$ and~$k[x,y,y^{-1}]$. Now we have to find functions that are defined on both open sets, but this is just~$k[x,y]$! Hence we find~$\mathcal{O}(U)=k[x,y]$, which is clearly distinct from what we had in the affine scheme scenario: the ring of global sections is \emph{not} equal to~$A$ in~$\Spec A$ (even more: there is no ring~$A$ such that~$U=\Spec A$).
\end{example}

\begin{example}
  Now we'll introduce the \emph{projective line}~$\mathbb{P}^1_k$ and the \emph{affine line with the doubled origin}. These are both obtained by taking~$\mathbb{A}^1_k$ and gluing it in two different ways. Recall that gluing topological spaces is done by taking their disjoint union and identifying certain open sets (and their subsets).

  Now let~$X=\mathbb{A}^1_k\sqcup\mathbb{A}^1_k$. Take the open set~$\distinguished(t_1)$ in the first component and the open set~$\distinguished(t_2)$ in the second. Visualize this by drawing two parallel lines and drawing a dot at the origin if you like. Now identify these sets by the assignment~$t_1\mapsto t_2$ (and~$t_2\mapsto t_1$). We get almost the affine line, but the closed points corresponding to~$(t_1)$ and~$(t_2)$ are still unidentified.

  Another way to glue the two open sets is by~$t_1\mapsto 1/t_2$ (and likewise~$t_2\mapsto 1/t_1$). The origin in the first component is somehow sent to a point at infinity (or better: \emph{the} point) in the second component, while the origin in the second is sent to the point at infinity of the first. We've obtained the projective line in a rather ad-hoc way. I will discuss a more canonical method later.

  This should feel familiar, from our manifolds course. But whereas differential geometry excludes the affine line with the doubled origin (it is not Hausdorff enough for those pesky non-algebraic geometers) it is a scheme! It's not a scheme you will wish to work (we do like some notion of \emph{separated}) with but with the Zariski topology being so coarse (it's the cofinite topology for god's sake, you can't get much coarser) there's no way of excluding it: only trivial schemes are Hausdorff.
\end{example}

\begin{example}
  Now consider~$\Spec\mathbb{Z}[x]$. This is an affine scheme, so it might've been better to discuss it in my previous post, but I will leave this as an exercise: try to understand \href{http://www.neverendingbooks.org/index.php/mumfords-treasure-map.html}{Mumford's treasure map}.
\end{example}

Before I introduce projective schemes, a small word on local rings. Recall that for a domain there is a corresponding quotient field. Now we can consider a subring of the quotient field that is almost a domain: it has a unique maximal ideal. In the geometric sense this maximal ideal consists of the functions vanishing at a point. This geometric intuition is most familiar from manifolds, but applies to algebraic geometry too.

It is possible to localize at a certain prime ideal (or more precisely: the complement). So we will define the \emph{local ring at a point~$[\mathfrak{p}]$} to be the ring~$A_{\mathfrak{p}}$, where~$A$ is the ring of global sections of an affine open subset of our scheme. Convince yourself that this definition is correct.

One last term (I promise!) is the notion of a \emph{stalk at a point~$[\mathfrak{p}]$}. This is defined as the direct limit over all sections over open sets containing~$[\mathfrak{p}]$. Don't worry if this means nothing, I've \emph{defined} stalks to be the local rings they should be, but actually you should do it the other way around, making clear that the direct limit is this local ring. We've done this construction in our Commutative algebra course, but it was rather fuzzy at that moment (at least to me).

Now a scheme is said to be a \emph{locally ringed space}, it's stalks are local rings. Notice that this is one (hopefully) the most ambiguous terminology you'll ever encounter: the space is not ringed locally, it is equipped with local rings. It is therefore not some local property, it applies to all points. If you speak Dutch: it would be the difference between \emph{lokaalgeringde ruimte} and \emph{lokaal geringde ruimte}, we are using the first spelling.
