\section{What are schemes?}

Or phrased more strongly: why the heck do we need them? I was perfectly happy with varieties!

Well, you might remember the following relationship from the first lecture of \#angs@t (or from our algebraic geometry course last year):
\begin{equation}
  \text{affine varieties} \underset{\text{equivalence}}{\overset{\text{anti-}}{\longleftrightarrow}} \text{reduced and finitely generated $\mathbb{C}$-algebras}
  \label{equation:variety-anti-equivalence}
\end{equation}
with the anti-equivalence being expressed by taking~$\Spec$ and the corresponding coordinate ring.

The objects on the left are quite nice: we can easily make drawings of them (or at least of their counterparts in~$\mathbb{R}$), but to the right we find a set of (although nice) objects with severe restrictions.

We observe three mutually unrelated conditions you might want to relax:
\begin{enumerate}
  \item the absence of nilpotent elements.
  \item finitely generated;
  \item only defined over~$\mathbb{C}$;
\end{enumerate}
As for now, we don't want to lose the fact that we're dealing with algebras (or rings).

By dropping these restrictions we wish to gain
\begin{itemize}
  \item elegance of exposition;
  \item the bliss of stating things in their most general form;
  \item a tool and a language when you have to work with objects that do not meet this stringent conditions (number theory will benefit greatly from this).
\end{itemize}

These are all rather vague and abstract reasons. Geometrically speaking, there are at least three reasons to introduce schemes into the picture. First of all we would like to have an intrinsic notion of variety, independent of any embedding in projective space. This is comparable to the notion of differentiable manifold. Even though the Whitney embedding theorem states that you can embed every such manifold into an $\mathbb{R}^{n}$ for sufficiently high~$n$, the power of the intrinsic definition lies in getting rid of redundant information, and study the manifold in itself. Weil already defined abstract varieties in the forties which was the first step to this intrinsic notion, but it wasn't until schemes came along that the concept truly flourished.

The second reason comes from groups acting on projective varieties, for which the quotient isn't a variety. Locally though, it is, which indicates that there should be a deeper connection present. With the introduction of schemes, a lot of these examples could be thoroughly studied, which gave rise to Mumford's geometric invariant theory.

Finally, if you study families of algebraic varieties, some of the fibres (i.e., objects corresponding to a certain parameter) aren't varieties at all. Considering the family of all conics for example, we get a strange conic when we take~$X^{2}=0$, which corresponds set-theoretically to the straight line~$X=0$. This is obviously not the right way to think about the conic; just look at the parametrization~$X^{2}-t=0$. These are all genuine conics for all values of~$t$, except for $t=0$, indicating that something went wrong. Issues like these get nicely resolved by using schemes. 

Before I introduce scheme theory by giving the generalization of~\eqref{equation:variety-anti-equivalence} I'll discuss what happens if we drop each of the conditions:
\begin{description}
  \item[\textbf{reducedness}] by allowing nilpotents we gain objects like the dual numbers that allow a really nice build-up of tangent spaces in algebraic geometry and for instance intersection theory (the study of how curves and higher-dimensional objects intersect, counting intersections with multiplicities etc.) now becomes an essential part of our whole new language of schemes;
  \item[\textbf{finitely generated}] we'll get some nice counterexamples from dropping this condition but I'm not really familiar with any important consequences of this (it's like Noetherian rings, you know some that are not but most of them are);
  \item[\textbf{$\mathbb{C}$-algebras}] we have been working with curves over \emph{finite fields} from the beginning of this seminar and you might want to work over \emph{no field at all} too, for instance if you want to give a meaning to~$\Spec\mathbb{Z}[x]$;
\end{description}

So by analogy we now \emph{define} (you might remember this anti-equivalence from that same first lecture)
\begin{equation}
  \text{affine schemes} \underset{\text{equivalence}}{\overset{\text{anti-}}{\longleftrightarrow}} \text{commutative and unital rings}.
  \label{equation:scheme-anti-equivalence}
\end{equation}

It is not immediately obvious why the right-hand side is the correct choice, but you've got to admit it looks like a very nice generalization. What this implies for our geometrical intuition for objects on the left is less clear.

I will try to back up this definition and develop some intuition for this in the upcoming posts, but for a real dive into examples of schemes I refer you to \cite{geometry-of-schemes}, \cite{foag}, \cite{red-book} or if you have a certain masochistic tendency \cite{hartshorne}. My favourite is \cite{foag} by the way, it has pretty pictures and it tells a lot about number-theoretic schemes too.

Up next: the complete build-up to affine schemes.
