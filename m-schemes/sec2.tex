\section{What are schemes?}

Or phrased more strongly: why the heck do we need them? I was perfectly happy with varieties!

Well, you might remember the following relationship from the first lecture of \#angs@t (or from our algebraic geometry course last year):
\begin{equation}
  \text{affine varieties} \underset{\text{equivalence}}{\overset{\text{anti-}}{\longleftrightarrow}} \text{reduced and finitely generated $\mathbb{C}$-algebras}
  \label{equation:variety-anti-equivalence}
\end{equation}
with the anti-equivalence being expressed by taking~$\Spec$ and the corresponding coordinate ring.

The objects on the left are quite nice: we can easily make drawings of them (or at least of their counterparts in~$\mathbb{R}$), but to the right we find a set of (although nice) objects with severe restrictions.

We observe three mutually unrelated conditions you might want to relax:
\begin{enumerate}
  \item the absence of nilpotent elements.
  \item finitely generated;
  \item only defined over~$\mathbb{C}$;
\end{enumerate}
As for now, we don't want to lose the fact that we're dealing with algebras (or rings).

By dropping these restrictions we wish to gain
\begin{itemize}
  \item elegance of exposition;
  \item the bliss of stating things in their most general form;
  \item a tool and a language when you have to work with objects that do not meet this stringent conditions (if you're a number theorist for instance, and we all are a bit for the course of this seminar).
\end{itemize}

Before I introduce scheme theory by giving the generalization of~\eqref{equation:variety-anti-equivalence} I'll discuss what happens if we drop each of the conditions:
\begin{description}
  \item[\textbf{reducedness}] by allowing nilpotents we gain objects like the dual numbers which we've already used in our varieties course and for instance intersection theory now becomes an essential part of our whole new language of schemes;
  \item[\textbf{finitely generated}] we'll get some nice counterexamples from dropping this condition but I'm not really familiar with any important consequences of this (it's like Noetherian rings, you know some that are not but most of them are);
  \item[\textbf{$\mathbb{C}$-algebras}] we have been working with curves over \emph{finite fields} from the beginning of this seminar and you might want to work over \emph{no field at all} too, for instance if you want to give a meaning to~$\Spec\mathbb{Z}[x]$;
\end{description}

So by analogy we now \emph{define} (you might remember this anti-equivalence from that same first lecture)
\begin{equation}
  \text{affine schemes} \underset{\text{equivalence}}{\overset{\text{anti-}}{\longleftrightarrow}} \text{commutative and unital rings}.
  \label{equation:scheme-anti-equivalence}
\end{equation}

It is not immediately obvious why the right-hand side is the correct choice, but you've got to admit it looks like a very nice generalization. What this implies for our geometrical intuition for objects on the left is less clear.

I will try to back up this definition and develop some intuition for this in the upcoming posts, but for a real dive into examples of schemes I refer you to \cite{geometry-of-schemes}, \cite{foag}, \cite{red-book} or if you have a certain masochistic tendency \cite{hartshorne}. My favourite is \cite{foag} by the way, it has pretty pictures and it tells a lot about number-theoretic schemes too.

Up next: the complete build-up to affine schemes.
