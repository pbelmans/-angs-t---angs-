\section{The Mason-Stothers theorem and a history of both}
\label{section:mason-stothers}

\subsection{The Mason-Stothers theorem}

Enough conjectural stuff, time for some \emph{facts}. Did you know~$\mathbb{Z}$ is an Euclidean domain?

\begin{definition}
  An \emph{Euclidean domain} is an integral domain~$R$ equipped with an \emph{Euclidean function}~$d\colon R\setminus\left\{ 0 \right\}\to\mathbb{N}$ such that for~$a$ and~$b$ in~$R$ with~$b\neq 0$ we have a division with a remainder, i.e., there exist~$q$ and~$r$ in~$R$ such that
  \begin{equation}
    a=bq+r
  \end{equation}
  and either~$r=0$ or~$d(r)<d(b)$. An additional requirement could be~$d(a)\leq d(ab)$ for nonzero~$a$ and~$b$ but it can be proved that every Euclidean function can be modified in order to satisfy this extra property.
\end{definition}

\begin{remark}
  An Euclidean domain can be equipped with \emph{several} Euclidean functions. When there exists a single function we can call~$R$ Euclidean, but all scalar multiples suffice too for instance.
\end{remark}

But~$\mathbb{Z}$ is not the only Euclidean domain. The polynomials~$k[X]$ over a field~$k$ can be used too! Now take~$k$ to be an algebraically closed field of characteristic~$0$. This will be necessary: we'll split polynomials into linear factors and we'll use the first derivative. This will break in non-algebraically closed fields or fields with characteristic different from~$0$. What happens in~$\mathbb{F}_1$ is left as a trivial exercise for the interested reader.

\begin{theorem}[Mason-Stothers]
  \label{theorem:mason-stothers}
  Let~$a(X)$, $b(X)$ and~$c(X)$ be three coprime polynomials (i.e., not sharing a common factor) such that
  \begin{equation}
    \label{equation:mason-stothers-equality}
    a(X)+b(X)=c(X)
  \end{equation}
  then there are \emph{no} triples~$(a(X),b(X),c(X))\in k[X]^3$ such that
  \begin{equation}
    \max\left\{ \deg a(X),\deg b(X),\deg c(X) \right\}>\deg\rad(abc(X))-1.
  \end{equation}
\end{theorem}

The function~$\rad\colon k[X]\to k[X]$ now maps a polynomial~$f(X)$ to the polynomial of minimal degree such that it has the same roots as~$f$. Under the assumption of an algebraically closed field this is the same as removing the multiplicities of the linear terms. Can you see the analogy?

\begin{remark}
  My statement of the~$abc$\nobreakdash-conjecture was phrased in terms non-negative integers, removing the need for a maximum over the values of the Euclidean function~$|\cdot|$. The original post introducing the~$abc$\nobreakdash-conjecture states the inequality in this more general form. In the statement of the Mason-Stothers theorem the presence of an Euclidean function is more obvious, so let's phrase the $abc$-conjecture in the Euclidean way for completeness' sake:
\end{remark}

\begin{conjecture}
  Let~$a$,~$b$ and~$c$ denote coprime integers (i.e., in~$\mathbb{Z}$) such that
  \begin{equation}
    a+b=c
  \end{equation}
  then for~$\epsilon>0$ there are only finitely many triples~$(a,b,c)\in\mathbb{Z}^3$ such that
  \begin{equation}
    \max\left\{ |a|,|b|,|c| \right\}>\rad(abc)^{1+\epsilon}.
  \end{equation}
\end{conjecture}

But have you noticed it? The statement of the Mason-Stothers theorem doesn't require an~$\epsilon$ and there are not finitely many counterexamples: there simply are \emph{no} counterexamples to the inequality! See \iftex Section~\ref{section:proof-and-consequences}\fi\ifblog my next post\fi for an actual proof. But first I want to throw some dates at you.

\subsection{History}

For the actual overview I need a preliminary remark, restating the~$abc$\nobreakdash-conjecture.
\begin{remark}
  \label{remark:abc-with-factor}
  We could also phrase the~$abc$\nobreakdash-conjecture in terms of a constant~$K_{\epsilon}$. In that case we have for~$\epsilon>0$ the existence of a~$K_{\epsilon}>1$ such that
  \begin{equation}
    c\leq K_{\epsilon}\rad(abc)^{1+\epsilon}.
  \end{equation}
  Hence the constant~$K_{\epsilon}$ swallows all finitely many counterexamples of the inequality \iftex in~\eqref{equation:abc-inequality}\fi\ifblog$c>\rad(abc)^{1+\epsilon}$\fi.
\end{remark}

\begin{description}
  \item[\textbf{1981}] Wilson Stothers states the theorem in his paper Polynomial identities and Hauptmoduln but the theorem is not remarked upon.
  \item[\textbf{1983}] R.\ Mason rediscovers the theorem and publishes it in his book Diophantine Equations over Function Fields.
  \item[\textbf{1985}] Joseph Oesterl\'e and David Masser conjecture what is now known as the~$abc$\nobreakdash-conjecture but might as well be (and sometimes is) called the Oesterl\'e-Masser conjecture.

  \item[\textbf{1986}] A first upper bound of~$c$ in terms of~$\rad(abc)$ is given by Stewart and Tijdeman. They proved the existence of a computable constant~$K_{\epsilon}$ such that
    \begin{equation}
      z<\exp\left( K_{\epsilon}\rad(abc)^{15} \right).
    \end{equation}
  \item[\textbf{1991}] Stewart and Yu improve the upper bound to
    \begin{equation}
      c<\exp\left( \rad(abc)^{2/3+K_{\epsilon}/\log\log\rad(abc)} \right)
    \end{equation}
    with~$K_{\epsilon}$ a computable constant.
  \item[\textbf{2000}] Noah Snyder, while still in high school, gives an elementary proof of the Mason-Stothers theorem in the paper An alternate proof of Mason's theorem, which I will reproduce in \iftex Section~\ref{section:proof-and-consequences}\fi\ifblog my next post\fi.
  \item[\textbf{2001}] Stewart and Yu again improve their bound, proving both
    \begin{equation}
      c<\exp\left( K_{\epsilon}\rad(abc)^{1/3}(\log\rad(abc))^3 \right).
    \end{equation}
    and
    \begin{equation}
      c<\exp\left( K_{\epsilon}\rad(abc)^{1/3+\epsilon} \right).
    \end{equation}
  \item[\textbf{2006}] ABC@Home begins its search.

  \item[\textbf{2007}] G\"yory and Yu improve a new bound on~$c$:
    \begin{equation}
      c<\exp\left( \frac{2^{10n+22}}{n^{n-4}}\rad(abc)(\log\rad(abc))^n \right)
    \end{equation}
    where~$n$ denotes the number of prime factors in~$abc$.
  \item[\textbf{2011}] We have a seminar about it, relating it to geometry over~$\mathbb{F}_1$.
\end{description}
