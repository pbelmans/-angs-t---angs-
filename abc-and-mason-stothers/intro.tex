\section*{Introduction}

As my first attempt to student participation in the {\iftex\#\fi\ifblog#\fi}angs@t / angs+ seminar, I decided to tell something about the the~$abc$\nobreakdash-conjecture and the related Mason-Stothers theorem. The first has been hinted at but not yet formally introduced and the second is a statement for polynomials that is analogous to the first. And more importantly: it is a \emph{real theorem}, with an easy proof. I will introduce both, give facts about both and I will prove \sout{both} \emph{the latter}.

\iftex
The outline: in Section~\ref{section:abc} I will first give the formal statement of the $abc$-conjecture, it is already given online in \href{http://www.noncommutative.org/index.php/the-abc-conjecture.html}{the post of the same name} but we haven't seen it yet in the seminar. For personal reference I'll repeat this, together with some generalities about it. To end the section I discuss some consequences.
\fi
\ifblog
The outline: in this first post I will first give the formal statement of the $abc$-conjecture, it is already given online in \href{http://www.noncommutative.org/index.php/the-abc-conjecture.html}{the post of the same name} but we haven't seen it yet in the seminar. For personal reference I'll repeat this, together with some generalities about it. To end the section I discuss some consequences.
\fi

\iftex
Then in Section~\ref{section:mason-stothers} I will talk about the analogy for polynomials, resulting in the Mason-Stothers theorem. Afterwards I'll discuss the history of both the Mason-Stothers theorem and the~$abc$\nobreakdash-conjecture.
\fi
\ifblog
In the second post I will talk about the analogy for polynomials, resulting in the Mason-Stothers theorem. Afterwards I'll discuss the history of both the Mason-Stothers theorem and the~$abc$\nobreakdash-conjecture.
\fi

\iftex
To end my attempt I will give in Section~\ref{section:proof-and-consequences} an easy proof of the Mason-Stothers theorem and to conclude I will give some consequences, mainly the analogon of Fermat's last theorem for polynomials. The three big parts of this will be written as three separate posts, each is intended as material for a lecture of half an hour, somewhere in the seminar.
\fi
\ifblog
To end my attempt I will give in the third post an easy proof of the Mason-Stothers theorem and to conclude I will give some consequences, mainly the analogon of Fermat's last theorem for polynomials. The two post that follow this one will be put online when they are ready.
\fi
