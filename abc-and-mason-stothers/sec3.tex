\section{Proof of the Mason-Stothers theorem and some consequences}
\label{section:proof-and-consequences}

\subsection{The proof}

Consider the statement of \iftex Theorem~\ref{theorem:mason-stothers}\fi\ifblog the Mason-Stothers theorem as in my previous post\fi. We'll prove some lemmas first, in which we'll denote the derivative of a polynomial~$f(X)$ by $f'(X)$. And we assume to be working in characteristic zero.

The first is a well-known fact from one of our algebra courses, but for completeness' sake I repeat it:

\begin{lemma}
  Let~$f(X)\in k[X]$ then~$f$ has a multiple root if and only if~$f(X)$ and~$f'(X)$ have a common root.

  \begin{proof}
    If~$f$ has a root~$\alpha$ of multiplicity~$n$ it can be written as~$(X-\alpha)^ng(X)$ where~$g(\alpha)\neq 0$. Now~$f'(X)$ is given by
    \begin{equation}
      n(X-\alpha)^{n-1}g(X)+(X-\alpha)^ng'(X)=(X-\alpha)^{n-1}\left( ng(X)+(X-\alpha)g(X) \right).
    \end{equation}
    If~$n=1$ evaluating~$f'(X)$ in~$\alpha$ yields~$ng(\alpha)\neq 0$ as we've assumed characteristic~$0$. Likewise~$n\geq 2$ yields~$f'(\alpha)=0$ because~$(X-\alpha)^{n-1}$ doesn't vanish.
  \end{proof}
\end{lemma}

Now the second lemma, which is a nice statement about the number of roots when they are not necessarily distinct.

\begin{lemma}
  \label{lemma:number-of-roots}
  Let~$f(X)\in k[X]$ then we have
  \begin{equation}
    \deg\gcd(f(X),f'(X))=\deg f(X)-\deg\rad f(X).
  \end{equation}

  \begin{proof}
    Because we've assumed that~$k$ is algebraically closed we can write~$f(X)$ as
    \begin{equation}
      \prod_{i=1}^m(X-\alpha_i)^{n_i}
    \end{equation}
    and~$\sum_{i=1}^mn_i=\deg f$. We now obtain from our previous lemma that
    \begin{equation}
      \gcd(f(X),f'(X))=\prod_{i=1}^m(X-\alpha_i)^{n_i-1}
    \end{equation}
    so
    \begin{equation}
      \begin{aligned}
        \deg\gcd(f(X),f'(X))&=\sum_{i=1}^m(n_i-1) \\
        &=\left( \sum_{i=1}^mn_i \right)-m \\
        &=\deg f(X)-m \\
        &=\deg f(X)-\deg\rad f(X)
      \end{aligned}
    \end{equation}
    as~$\rad f(X)$ removes the multiplicities from the linear factors.
  \end{proof}
\end{lemma}

We are now ready for the actual proof. No need to develop an intersection theory on~$\mathbb{P}^1/\mathbb{F}_1\times\operatorname{Spec}\mathbb{Z}$, only some polynomial manipulation.

\begin{proof}[Proof of the Mason-Stothers theorem]
  We have
  \begin{equation}
    \label{equation:key-equality}
    \begin{aligned}
      a'(X)b(X)-a(X)b'(X)&=a'(X)\left( c(X)-a(X) \right)-a(X)\left( c'(X)-a'(X) \right) \\
      &=a'(X)c(X)-a(X)c'(X)
    \end{aligned}
  \end{equation}
  by the linearity of the derivative.

  Now at least two of our chosen polynomials must be non-constant, otherwise the third is constant too, \iftex by~\eqref{equation:mason-stothers-equality}\fi\ifblog because~$a(X)+b(X)=c(X)$\fi. Assume~$a(X)$ and~$b(X)$ are non-constant, we obtain~$a'(X)b(X)-a(X)b'(X)\neq 0$ because if~$a'(X)b(X)=a(X)b'(X)\neq 0$ implies~$b\,|\,b'$ which gives us a contradiction.

  We now have that~$\gcd(a(X),a'(X))$ and~$\gcd(b(X),b'(X))$ both divide the left-hand side~\eqref{equation:key-equality} while~$\gcd(c(X),c'(X))$ divides the right-hand side. As~$a(X)$,~$b(X)$ and~$c(X)$ we obtain
  \begin{equation}
    \gcd\left( a(X),a'(X) \right)\gcd\left( b(X),b'(X) \right)\gcd\left( c(X),c'(X) \right)\,|\,a'(X)b(X)-a(X)b'(X).
  \end{equation}

  By applying~$\deg$ to this statement we obtain
  \begin{equation}
    \begin{gathered}
      \deg\gcd\left( a(X),a'(X) \right)+\deg\gcd\left( b(X),b'(X) \right)+\deg\gcd\left( c(X),c'(X) \right) \\
      \leq\deg\left( a'(X)b(X)-a(X)b'(X) \right)
    \end{gathered}
  \end{equation}
  hence
  \begin{equation}
    \label{equation:gcd-inequality}
    \begin{gathered}
      \deg\gcd\left( a(X),a'(X) \right)+\deg\gcd\left( b(X),b'(X) \right)+\deg\gcd\left( c(X),c'(X) \right) \\
      \leq\deg a(X)+\deg b(X)-1
    \end{gathered}
  \end{equation}
  because~$\deg a'(X)b(X)-a(X)b'(X)$ has at most degree~$\deg a(X)+\deg b(X)-1$ which is the case when no cancellation occurs: both terms obviously have degree~$\deg a(X)+\deg b(X)-1$ where the minus one comes from the derivative (yet again we need characteristic zero).

  We can now invoke Lemma~\ref{lemma:number-of-roots}, obtaining
  \begin{equation}
    \begin{aligned}
      \deg\gcd\left( a(X),a'(X) \right)&=\deg a(X)-\deg\rad a(X) \\
      \deg\gcd\left( b(X),b'(X) \right)&=\deg b(X)-\deg\rad b(X) \\
      \deg\gcd\left( c(X),c'(X) \right)&=\deg c(X)-\deg\rad c(X)
    \end{aligned}
  \end{equation}

  If we plug these values in~\eqref{equation:gcd-inequality} we obtain
  \begin{equation}
    \begin{aligned}
      \deg c(X)&\leq\deg\rad a(X)+\deg\rad b(X)+\deg\rad c(X)-1\\
      &=\deg\rad abc(X)-1.
    \end{aligned}
  \end{equation}

\end{proof}

Remark that we really needed the fact that it's an algebraically closed field of characteristic zero for this simple proof. It is possible to alter the proof for non-algebraically closed fields of characteristic zero.

\subsection{Consequences}
Just like the $abc$\nobreakdash-conjecture for~$\mathbb{Z}$ implied the Fermat-Wiles theorem (this is a funny composition of conjecture, implication and theorem) for sufficiently big exponents, the Mason-Stothers theorem implies a polynomial version of Fermat's last theorem.

\begin{theorem}[Fermat's theorem for polynomials]
  \label{theorem:polynomial-fermat}
  Let~$n\geq 3$. There are no solutions to the equation
  \begin{equation}
    u^n(X)+v^n(X)=w^n(X)
    \label{equation:polynomial-fermat}
  \end{equation}
  where~$u(X),v(X)$ and~$w(X)\in k[x]$ are relatively prime.

  \begin{proof}
    Take~$a(X)=u^n(X)$, $b(X)=v^n(X)$ and~$c(X)=w^n(X)$. By the Mason-Stothers theorem we have
    \begin{equation}
      \deg u^n(X)\leq\deg\rad(u^nv^nw^n(X))-1.
    \end{equation}
    but as~$\deg u^n(X)=n\deg u(X)$ and~$\deg\rad(u^n(X))=\deg\rad(u(X))\leq\deg u(X)$ we obtain
    \begin{equation}
      n\deg u(X)\leq\deg u(X)+\deg v(X)+\deg w(X)-1.
    \end{equation}

    Analogously we have
    \begin{equation}
      \begin{gathered}
        n\deg v(X)\leq\deg u(X)+\deg v(X)+\deg w(X)-1 \\
        n\deg w(X)\leq\deg u(X)+\deg v(X)+\deg w(X)-1
      \end{gathered}
    \end{equation}
    hence combining these yields
    \begin{equation}
      n\deg uvw(X)\leq 3\deg uvw(X)-3<3\deg uvw(X)
    \end{equation}
    which yields~$n<3$ (unless the degree is zero), as the Mason-Stothers theorem must be satisfied. The condition it provides on the degrees of relatively prime polynomials provides the nonexistence of solutions to \eqref{equation:polynomial-fermat}.
  \end{proof}
\end{theorem}

For other consequences, you could try the following as exercises (they are taken from \url{http://topologicalmusings.wordpress.com} and Lang's Undergraduate Algebra):

\begin{theorem}[Davenport's theorem]
  Let~$u(X)$ and~$v(X)$ be two non-constant relatively prime polynomials such that~$u^3(X)-v^2(X)\neq 0$. Show that
  \begin{equation}
    \begin{aligned}
      \frac{1}{2}\deg u(X)&\leq\deg(u^3(X)-v^2(X))-1 \\
      \frac{1}{3}\deg v(X)&\leq\deg(u^3(X)-v^2(X))-1
    \end{aligned}.
  \end{equation}
\end{theorem}

\begin{theorem}
  Let~$u(X)$ and~$v(X)$ be two polynomials over~$\mathbb{C}$ and~$\alpha\in\mathbb{C}\setminus\left\{ 0 \right\}$. If
  \begin{equation}
    u^3(X)=v^2(X)+\alpha
  \end{equation}
  then~$u(X)$ and~$v(X)$ are constants.
\end{theorem}
