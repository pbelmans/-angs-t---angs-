\section{Introduction}

In order to understand some of the approaches to a geometry over the field with one element (as outlined in \cite{mapping-fun}) one needs to be sufficiently fluent in the language of schemes. This comes as no surprise, schemes being the foundation blocks of modern algebraic geometry. Therefore an obvious way of inventing a geometry over~$\mathbb{F}_1$ is translating the language of schemes to something more general.

Because scheme theory is not a part of the regular B.Sc.\ curriculum and due to popular demand, I will write a small primer in scheme theory and~$\mathcal{M}$-scheme theory, trying to compromise three goal:
\begin{enumerate}
  \item introduce schemes formally, together with some examples;
  \item lay bridges to our knowledge of algebraic geometry using varieties and introduce similar concepts in other fields of mathematics;
  \item apply our newly gained knowledge of schemes to introduce~$\mathcal{M}$-schemes, the moniker used in \cite{mapping-fun} to the approach of Kato and Deitmar, trying to understand why and how this relates to~$\mathbb{F}_1$.
\end{enumerate}

In what follows I will assume some basic knowledge about topology, varieties, categories and differential geometry. In the second post of the series (i.e., including this introduction) I will talk about why we wish to introduce the notion of a scheme and how it relates to what we already know from algebraic geometry with varieties. The next post will be the build-up of affine schemes, giving all the basic ideas we'll need. Then comes a post about schemes in general and projective schemes.

The fifth post in the series will be a summary of \cite{kapranov-smirnov}, discussing what we'd like from a geometry over~$\mathbb{F}_1$. The main goal of this post will be to find out how to define a geometry over~$\mathbb{F}_1$, finding certain peculiarities of~$\mathbb{F}_1$ we wish to incorporate in our theory. And when all that has been done, we are ready to introduce~$\mathcal{M}$-schemes. As will become clear, this approach is the most scheme-theoretic of all. If I still feel like writing at that point, I will discuss the presence of schemes in the other approaches.
