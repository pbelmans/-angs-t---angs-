\section*{Introduction}

In order to understand some (or most) of the approaches to a geometry over the field with one element (as outlined in \cite{mapping-fun}) one needs to be sufficiently fluent in the language of schemes. This comes as no surprise, schemes being the foundation blocks of modern algebraic geometry. Therefore an obvious way of inventing an algebraic geometry over~$\mathbb{F}_1$ is translating the language of schemes to something that looks a lot like it but is more general.

Because scheme theory is not a part of the regular B.Sc.\ curriculum (and due to popular demand), I will write a small primer in schemes leading to an introduction to~$\mathcal{M}$-schemes, trying to compromise three goals:
\begin{enumerate}
  \item introduce schemes formally, together with some examples;
  \item refer to our knowledge of algebraic geometry using varieties and relate scheme theory to other parts (i.e., not algebraic geometry) of mathematics;
  \item apply our newly gained knowledge of schemes to introduce~$\mathcal{M}$-schemes, the moniker used in \cite{mapping-fun} to the approach of Kato and Deitmar, trying to understand why and how this relates to (a geometry over)~$\mathbb{F}_1$.
\end{enumerate}

In what follows I will assume some basic knowledge about topology, varieties, categories and differential geometry. But whenever we haven't seen it explicitly in some mandatory course I will elaborate.

As this is an ambitious project, one post obviously won't suffice. This post serves both as an announcement and a discussion of why we wish to introduce the notion of a scheme and how it relates to what we already know from algebraic geometry with varieties. The next post will be the build-up of affine schemes, giving all the basic ideas we'll need. Then comes the third post discussing full-fledged scheme theory, also mentioning projective schemes.

The fourth post in the series will be a (partial) summary of \cite{kapranov-smirnov}, discussing what we'd like from a geometry over~$\mathbb{F}_1$. The main goal of that post will be to find out how to define a geometry over~$\mathbb{F}_1$, finding certain properties we wish to incorporate in our geometry. And when all that preparatory work has been done, we are ready to introduce~$\mathcal{M}$-schemes. As will become clear, this approach is the most scheme-theoretic of all (as discussed in \cite{mapping-fun}). If I still feel like writing at that point, I will write a sixth post in which I discuss how scheme theory arises in the other approaches.
