\section{Introduction}
In the seminar of October 28th a myriad of concepts have been introduced, relating them all to each other and our main object of interest: the ring of big Witt vectors~$\mathrm{W}(\mathbb{Z})$. This object will arise in Borger's approach of a geometry over~$\mathbb{F}_1$ and in order to understand this complex structure we'd like to see as many different faces of it as possible.

The goal of that installment of the seminar was to discuss (the relevant parts of) the paper~\cite{the-burnside-ring}, which introduces four different incarnations of~$\mathrm{W}(\mathbb{Z})$ and some interesting combinatorial relations between them. Due to exhaustion of the public we didn't get all the way to the end. As the most combinatorics-minded person attending I took the task of tying up the loose ends upon myself.

Everything we'd like to make sense of is given in the following commutative diagram.

\begin{equation}
  \begin{array}{ccccc}
    & & \mathrm{Nr}(\mathbb{Z}) & & \\
    & & \phantom{\textrm{itp}}\big\downarrow\textrm{itp} & & \\
    \mathrm{W}(\mathbb{Z}) & \overset{\tau}{\longrightarrow} & \hat{\Omega}(\mathrm{C}_\infty) & \overset{\mathrm{s}_t}{\longrightarrow} & \Lambda(\mathbb{Z}) \\
    \phantom{\Phi}\big\downarrow\Phi & & \phantom{\hat{\varphi}}\big\downarrow\hat{\varphi} & & \phantom{\mathrm{L}_{\mathbb{Z}}}\big\downarrow\mathrm{L}_{\mathbb{Z}} \\
    \prod_{n=1}^{+\infty}\mathbb{Z} & \longleftrightarrow & \mathrm{g}(\mathrm{C}_\infty)=\mathbb{Z}^{\mathbb{N}^\times} & \longleftrightarrow & t\mathbb{Z}[\![t]\!]
  \end{array}.
\end{equation}

I will explain all objects and their maps and prove the isomorphisms of the ring of big Witt vectors~$\mathrm{W}(\mathbb{Z})$, the Burnside ring of the infinite cyclic group~$\hat{\Omega}(\mathrm{C}_\infty)$, the necklace algebra~$\mathrm{Nr}(\mathbb{Z})$ and the~$\lambda$\nobreakdash-ring~$\Lambda(\mathbb{Z})$. The objects on the bottom of the diagram are well-known and their maps are the obvious ones: coefficients of a power series~$\sum_{n=1}^{+\infty}a_nt^n$ in~$t\mathbb{Z}[\![t]\!]$ are sent to their (infinite-dimensional) coefficient vector~$(a_n)_{n=1}^{+\infty}$ and the maps~$f\colon\mathbb{N}\to\mathbb{Z}$ such that~$f(n)=a_{n}$.

In the seminar the construction of~$\Lambda(\mathbb{Z})$ has already been given thoroughly, so I will discuss is only briefly. On the other hand I will give much attention to understanding the Burnside ring of the infinite cyclic group~$\hat{\Omega}(\mathrm{C}_\infty)$, relating it to representation theory and telling something about its origins. The ring of big Witt vectors~$\mathrm{W}(\mathbb{Z})$ and the necklace algebra~$\mathrm{Nr}(\mathbb{Z})$ will be discussed in depth as well, just like their combinatorial relationships.

By doing so we will be able to switch back and forth between interpretations of~$\mathrm{W}(\mathbb{Z})$ in Borger's approach, using the incarnation suited for the task at hand.

This first \iftex section\fi\ifblog post\fi served as an announcement and introduction, in the second \iftex section\fi\ifblog post\fi I will introduce all objects in the commutative diagram. At the same time I will discuss the \emph{vertical} arrows. The third \iftex section\fi\ifblog post\fi will be devoted to a discussion of the \emph{horizontal arrows}~$\tau$ and~$\mathrm{s}_t$ (which are the interesting combinatorial isomorphisms) and some background to the structures we have discussed.
