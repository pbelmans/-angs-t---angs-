\section{Some background}
\subsection{Origins}
\begin{enumerate}
  \item topology on cyclic sets, closure etc.
\end{enumerate}

Now for the necklace algebra, the object with the most intriguing name of them all: the \emph{necklace polynomial} is defined to be
\begin{equation}
  \mathrm{M}(x,n)=\frac{1}{n}\sum_{d\divides n}\mu\left( \frac{n}{d} \right)x^d
\end{equation}
where~$\mu$ is the Moebius function. This numerical polynomial counts the number of \emph{primitive necklaces} of~$n$ beads, coloured using~$m$ distinct colours and asymmetric under rotation. Besides being interesting to tell your toddler it is related to the dimension of the piece of degree $n$ in the free Lie algebra on~$m$ generators (I'm not familiar with Lie theory but I suppose this makes sense to some of you) or the number of monic irreducible polynomials of degree~$n$ over a finite field where~$m$ is necessarily a prime power.
