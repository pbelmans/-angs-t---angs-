\usepackage[fleqn, leqno]{amsmath}
\usepackage{amsthm}
\usepackage[english]{babel}
\usepackage{booktabs}
\usepackage{hyperref}
\usepackage{hyphenat}
\usepackage{mathtools}
\usepackage{multirow}
\usepackage{tikz}
%\usepackage{titlesec}
\usepackage[normalem]{ulem}
\usepackage{todonotes}
\usepackage{wrapfig}
\selectlanguage{english}

\usepackage[T1]{fontenc}
\usepackage[charter]{mathdesign}
\usepackage[scaled]{beramono, berasans}
\usepackage[draft = false]{microtype}
\frenchspacing

\definecolor{uablue}{RGB}{0,61,100}
\definecolor{uared}{RGB}{126,0,47}
\definecolor{vividbrown}{RGB}{215,154,70}
\definecolor{uagreen}{RGB}{0,126,17}

\hypersetup{hypertexnames=false,draft=false,bookmarksdepth=1,bookmarksopen=true,colorlinks,linkcolor=uablue,citecolor=uablue,urlcolor=uared,pdfstartview={XYZ null null 1}}

\counterwithout{section}{chapter}
\counterwithin*{section}{chapter}
\setcounter{secnumdepth}{1}
\setcounter{tocdepth}{1}

\chapterstyle{companion}

\newif\ifblog
\newif\iftex
\blogfalse
\textrue

\addtolength{\parskip}{.7ex}
\relpenalty=10000
\binoppenalty=10000 
\setlength\parindent{0cm}

\addtolength\headheight{12pt}

\makeatletter
\renewcommand\@chapapp{Series}
\makeatother

\newcommand\myauthor[1]{Written by #1\par}

\theoremstyle{definition}
\newtheorem{theorem}{Theorem}
\newtheorem{conjecture}[theorem]{Conjecture}
\newtheorem{corollary}[theorem]{Corollary}
\newtheorem{definition}[theorem]{Definition}
\newtheorem{example}[theorem]{Example}
\newtheorem{lemma}[theorem]{Lemma}
\newtheorem{notation}[theorem]{Notation}
\newtheorem{proposition}[theorem]{Proposition}
\newtheorem{remark}[theorem]{Remark}

\newcommand\unityroots{\mathbf{\mu}}
\newcommand\divides{\mathbin{|}}

\DeclareMathOperator\Aut{Aut}
\DeclareMathOperator\character{char}
\DeclareMathOperator\distinguished{D}
\DeclareMathOperator\divisor{div}
\DeclareMathOperator\Frob{Frob}
\DeclareMathOperator\Gal{Gal}
\DeclareMathOperator\genus{g}
\DeclareMathOperator\GL{GL}
\DeclareMathOperator\id{id}
\DeclareMathOperator\identity{id}
\DeclareMathOperator\lcm{lcm}
\DeclareMathOperator\mex{mex}
\DeclareMathOperator\On{\mathsf{On}}
\DeclareMathOperator\ord{ord}
\DeclareMathOperator\Proj{Proj}
\DeclareMathOperator\quality{q}
\DeclareMathOperator\rad{rad}
\DeclareMathOperator\res{res}
\DeclareMathOperator\Spec{Spec}
\DeclareMathOperator\Stab{Stab}
\DeclareMathOperator\Supp{Supp}
\DeclareMathOperator\vanishing{V}
