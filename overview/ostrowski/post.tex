The intent of this post is show one of the striking similarities between the ring of integers~$\mathbb{Z}$, and the ring of polynomials~$k[T]$ with coefficients in a field~$k$.

As the theorem of Ostrowski shows, all the absolute values on~$\mathbb{Q}$ can be classified:
\begin{itemize}
  \item the archimedian absolute value, usually denoted~$|\cdot|_\infty$;
  \item The non-archimedian absolute values~$|\cdot|_p$, with~$p$ a prime number.
\end{itemize}

We recall that these last (ultra)norms are all determined by a \emph{valuation}~$v_p\colon \mathbb{Q} \to \mathbb{Z}~$, which measures the exponent~$n$ of an element~$\frac{a}{b}p^n \in \mathbb{Q}$. The norm can then be expressed as~$ |x|_p = \theta^{-v_p(x)}$, with the remark that~$0<\theta<1$.

It is however useful to set~$\theta = p^{-1}$, because then the following product formula will hold:
\begin{equation}
  \left( \prod_p |x |_p \right) \cdot |x|_\infty = 1.
\end{equation}

We will now try to establish the same result for the ring~$k[T]$. Suppose~$P \in k[T]$ an irreductible unitary polynomial of degree~$\geq 1$ (the analogy between the primes should be clear). Define for a polynomial~$F \in k[T]$:
\begin{equation}
  v_P(F)\coloneqq \sup \{ n \in \mathbb{N} \mid P^n \ \text{divides} \ F \} .
\end{equation}

The following calculations show that~$v_P$ has the good properties and can be extended to become a valuation function on the fraction field of~$k[T]$, which we will denote by~$K$:
\begin{itemize}
  \item $v_P(0) = \sup\{ n\in \mathbb{N} \mid P^n \ \text{divides } 0 \} = +\infty$.
  \item $v_P(F.F') = v_P(F) + v_P(F')$. We use the fact that~$P$ is irreductible, otherwise the equality may not hold (consider for example~$P = (T-1)^2, F = F' = (T - 1)$).
  \item $v_P(F + F') \geq \min(v_P(F), v_P(F'))$ because if~$P^n$ divides both~$F$ and~$F'$, then it will also divide their sum. That it is not an equality, shows the example~$P = (T - 1), F = T, F' = -1$.
\end{itemize}

Obviously, we can now extend the map~$v_P$ to the fraction field~$K$ of~$k[T]$ by defining~$v_P(\frac{F}{G})\coloneqq v_P(F) - v_P(G)$.

We will now try to classify the absolute values on~$K$. This will take several steps:
\begin{itemize}
  \item First we prove that an absolute value trivial on~$k$, must be non-archimedian.
  \item Secondly, we will treat the case where~$|T|<1$.
  \item Finally, we will investigate what happens when~$|T| > 1$.
\end{itemize}

Suppose~$|\cdot|$ an absolute value on~$K$, such that~$|\cdot|$ is trivial on~$k$ ($\forall a \in k^*: |a| = 1$). We want to show that~$|\cdot|$ must be non-archimedian. When~$k$ itself is a field of prime characteristic, this is the case for any absolute value (use the Frobenius morphism). Thus assume~$k$ of characteristic 0, and consider the following computation (with~$F,G \in K$):
\begin{equation}
  |F + G|^n = |(F+G)^n| = \left|\sum_{i=0}^n \dbinom{n}{i}F^i G^{n-i} \right| \leq \sum_{i=0}^n \left| F^i G^{n-i} \right|.
\end{equation}
From here we can see
\begin{equation}
  |F+G| \leq \sqrt[n]{\sum_{i=0}^n \left| F^i G^{n-i} \right|} \leq \sqrt[n]{(n+1)\max(|F|, |G|)^n} = \sqrt[n]{n+1} \max(|F|,|G|).
\end{equation}
If we let~$n$ take off to infinity, then we get the ultra-triangle inequality desired. So far, we have shown that if~$|\cdot|$ is trivial on~$k$, it must be non-archimedian.

Next up, we assume that~$|T|< 1$. The other case will be treated separately.
If~$F = \sum_{i=0}^n a_iT^i \in k[T]$, then~
\begin{equation}
  |F| \leq \max_i |a_iT^i| \leq 1.
\end{equation}
It can easily be shown that the set~$ \{ F \in k[T] \mid |F| < 1 \}~$ is an prime ideal of~$k[T]$. We may assume that the absolute value we are discussing is not completely trivial, so there will be an non-zero element with norm different from 1, and this implies the set not being 0. Therefore this ideal is generated by an irreducible unitary polynomial~$P$.

We can conclude that if a polynomial~$F$ is not divisible by~$P$, then it must have norm~$|F| = 1$. We will note~$R = |P| < 1$. The final thing we want to establish is that~
\begin{equation}
  |F| = R^{-v_P(F)}.
\end{equation}
Therefore we take~$F \in (P)$, and we can easily deduce that~$F$ is the product of~$P^{v_P(F)}$ and~$F'$, the last being a part not divisible by~$P$. We find
\begin{equation}
  |F| = |P|^{v_P(F)}\cdot |F'| = R^{-v_P(F)} .
\end{equation}
As a notation, we will use~$|\cdot|_P$.

We still have to treat the case were~$|T| > 1$. Since this absolute value still is non-archimedian, we can quickly conclude that~
\begin{equation}
  |F| = |a_0 + a_1 T + \ldots + a_n T^n| \leq \max_i |a_i|\cdot |T|^i = |T|^{\deg F} .
\end{equation}
We will denote this absolute value with~$|\cdot|_\infty$. At this point, we have proven something similar to Ostrowski: that every nontrivial absolute value on~$K$ which happens to be trivial on~$k$, is either equivalent with~$|\cdot|_\infty$ or with~$|\cdot|_P$ for some~$P$.

As a final remark, one can prove that the following product formula holds~$\forall F \in K^*:$
\begin{equation}
  \left( \prod_{P} |F|_P^{\deg(P)} \right) \cdot |F|_\infty =1 .
\end{equation}
The reason that the exponent~$\deg(P)$ shows up is that the absolute values can't be nicely normalized (remember in the number case we had to put~$\theta = p^{-1}$ to make it work).
