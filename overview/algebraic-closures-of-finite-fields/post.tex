My first post is going to be about the algebraic closure of a finite field~$\mathbb{F}_q$, where as usual~$q$ stands for a power of a prime~$p$. More exactly, I will describe what the Galois group~$\Gal(\overline{\mathbb{F}_q}/\mathbb{F}_q)$ is, what the subfields are of~$\overline{\mathbb{F}_q}$ and the Galois group~$\Gal(\overline{\mathbb{F}_q}/F)$ with~$F$ a field~$\mathbb{F}_q \subseteq F \subseteq \overline{\mathbb{F}_q}$.

\subsection{The algebraic closure and~$\Gal(\overline{\mathbb{F}_q}/\mathbb{F}_q)$}
First of all, what is the algebraic closure of~$\mathbb{F}_q$ ? The easiest way to describe this is
\begin{equation}
  \overline{\mathbb{F}_q} = \bigcup_{i=1}^n \mathbb{F}_{q^n}, 
\end{equation}
but there are other ways, like taking a direct limit, but luckily, they are all equal. It won't matter how you calculate $\Gal(\overline{\mathbb{F}_q}/\mathbb{F}_q)$, it's always going to be the same group (isomorphism not withstanding).

We know that $\Gal(\mathbb{F}_{q^n}/\mathbb{F}_q)$ is generated by the Frobenius automorphism~$\sigma(x) = x^q$, so $\Gal(\mathbb{F}_{q^n}/\mathbb{F}_q) = \mathbb{Z}/n\mathbb{Z}$. We also know that~$\mathbb{F}_{q^n} \subseteq \mathbb{F}_{q^m}$ iff~$n |m~$, so we get a directed system of finite extensions of~$\mathbb{F}_q$. We also get an inverse system of Galois groups.

\begin{definition}
  An inverse system of finite groups is a set of finite groups~$G_i$, indexed by a directed set~$I$ together with a set of group homomorphisms~$\phi_{ij}:G_j \rightarrow G_i$, where as usual~$\phi_{ij} \circ \phi_{jk} = \phi_{ik}$ when~$i \geq j \geq k$ and~$\phi_{ii} = 1_{G_i}$.
\end{definition}

This inverse system of Galois groups gives us the opportunity to take the projective limit
\begin{equation}
  \varprojlim_{n \geq 1} \Gal(\mathbb{F}_{q^n}/\mathbb{F}_q) = \{ (\sigma_{n})_{n \geq 1} \mid \sigma_m|_{\mathbb{F}_{q^n}} = \sigma_{n} \text{ if } n|m \} \subseteq \prod_{n \geq 1} \Gal(\mathbb{F}_{q^n}/\mathbb{F}_q). 
\end{equation}
Since~$\Gal(\mathbb{F}_{q^n}/\mathbb{F}_q) \cong\mathbb{Z}/n\mathbb{Z}$, this is nothing but~$\widehat{\mathbb{Z}}$, which in turn is isomorphic to~$\prod_{p_{i} \text{ prime}} \mathbb{Z}_{p_{i}}$ (this is easily proved using the Chinese Remainder Theorem). It is easy to see that~$\sigma:x \rightarrow x^q$ corresponds with~$(1 \bmod n)_{n \geq 1}$.

Now~$\widehat{\mathbb{Z}}$ has another nice property: if~$S$ is any finite subset of~$\mathbb{N}$, then
\begin{equation}
  \widehat{\mathbb{Z}} \subseteq \prod_{n \in \mathbb{N}-S} \mathbb{Z}/n\mathbb{Z}, 
\end{equation}
so deleting any finite number of numbers to take the inverse limit~$\varprojlim_{n \geq 1, n \notin S} \mathbb{Z}/n\mathbb{Z}$ doesn't change a thing.

\subsection{Steinitz numbers and subfields of~$\overline{\mathbb{F}_q}$}
Steinitz numbers will help us to find all subfields of~$\overline{\mathbb{F}_q}$, apart from the (obvious) finite ones.

\begin{definition}
  A Steinitz number is a symbol of the form
  \begin{equation}
    \prod_{p_i \text{prime}}{p_i}^{e_{i}}
  \end{equation}
  where~$p_{i}$ denotes the~$i$-th prime number and each~$e_i$ belongs to~$\mathbb{N} \cup \{\infty\}$. Two Steinitz numbers are equal iff for all~$p_i$ the corresponding exponents are the same. The set of Steinitz numbers is denoted~$\mathbb{E}$.
\end{definition}

We will use capital letters for abstract Steinitz numbers and lowercase letters for positive integers, who we will also treat as Steinitz numbers. For two Steinitz numbers~$N=\prod_{i=1}^\infty {p_i}^{e_{i}}$ en~$M=\prod_{i=1}^\infty {p_i}^{f_{i}}$, we define their product~$NM$ as the Steinitz number~$\prod_{i=1}^\infty {p_i}^{e_{i}+f_{i}}$ with the usual rules for addition with~$\infty$. We will say that a Steinitz number~$N$ divides a Steinitz number~$M$ iff for all~$i$,~$e_i \leq f_i$. In that case we define~$\frac{M}{N} = \prod_{i=1}^\infty {p_i}^{f_{i}-e_{i}}$, with the rule~$\infty - \infty = 0$. Otherwise,~$\frac{M}{N} $ is not defined. It is also not difficult to prove that~$N$ divides~$M$ iff for all~$n \in \mathbb{N}$ holds:~$n|N \Rightarrow n|M$. We also define the greatest common divisor~$N \wedge M= \prod_{i=1}^\infty {p_i}^{\min(f_{i},e_{i})}$ and the least common multiple~$N \vee M = \prod_{i=1}^\infty {p_i}^{\max(f_{i},e_{i})}$.

For a Steinitz number~$N$, we define~$$\mathbb{F}_{q^N} = \bigcup_{d \in \mathbb{N},d|N} \mathbb{F}_{q^d}.$$ It is easy to verify that this is a field (if~$n$ and~$m$ divide~$N$, then so does~$n \vee m$ and so the compositum~$\mathbb{F}_{q^n}\mathbb{F}_{q^m} = \mathbb{F}_{q^{n\vee m}}$ is a part of~$\mathbb{F}_{q^N}$, so the product of 2 elements is always defined, since all elements belong to finite fields). If~$n \in \mathbb{N}$, then~$\mathbb{F}_{q^n}$ is finite. It is also true that~$\mathbb{F}_{q^N} \subseteq \mathbb{F}_{q^M}$ iff~$N|M$. Moreover, one sees immediately that there is a bijection between Steinitz numbers and subfields of~$\overline{\mathbb{F}_{q}}$. We shall prove this.

\begin{theorem}
  There exists a bijection~$\phi\colon \mathbb{E} \rightarrow \{\text{ subfields of } \overline{\mathbb{F}_q} \}$ defined by~$\phi(N) = \mathbb{F}_{q^N}$.
  \begin{proof}
    To see that~$\phi$ is injective, take 2 different Steinitz numbers~$N=\prod_{i=1}^\infty {p_i}^{e_{i}}$ and~$M=\prod_{i=1}^\infty {p_i}^{f_{i}}$. Since they are different, there exists a prime~$p_i$ so that~$e_i \neq f_i$, so we take~$e_i < f_i$. But then we have that~$\mathbb{F}_{q^{p^{e_i+1}}} \subseteq \mathbb{F}_{q^M}$, but~$\mathbb{F}_{q^{p^{e_i+1}}} \nsubseteq \mathbb{F}_{q^N}$. So they aren't equal and thus~$\phi$ is injective.

    To prove it is surjective, let~$F$ be a subfield of~$\overline{\mathbb{F}_{q}}$ that contains~$\mathbb{F}_q$. We look for a Steinitz number~$S$ so that~$F = \mathbb{F}_{q^S}$. For each prime~$p_i$, let~$e_i$ be the greatest exponent so that~$\mathbb{F}_{q^{p_i^{e_i}}} \subseteq F$. We let~$e_i$ be~$\infty$ if this is true for every~$e_i \in \mathbb{N}$. Let~$S$ be~$\prod_{p_i \text{prime}} p_i^{e_i}$. First, it is easy to see that~$\mathbb{F}_{q^S}$ is the compositum of all~$\mathbb{F}_{p_i^{e_i}}$. Since all~$\mathbb{F}_{q^{p_i^{e_i}}}$ are subsets of~$F$, we have that~$\mathbb{F}_{q^S} \subset F$.
    
    For the reverse inclusion, take~$\alpha \in F$. Since~$\alpha$ is algebraic over~$\mathbb{F}_q$, it belongs to a certain~$\mathbb{F}_{q^n}$ with~$n \in \mathbb{N}$. If~$n = \prod_{p_i \text{prime}} p_i^{f_i}$, then~$f_i \leq e_i$ due to the maximality of~$e_i$. So~$n$ divides~$S$ and so~$\alpha \in S$. This concludes the proof.
  \end{proof}
\end{theorem}


One can calculate the Galois group~$\Gal(\overline{\mathbb{F}_q}/\mathbb{F}_{q^S})$ by using~$S$. It will come as no surprise that if~$S = \prod_{p_i \text{prime}} p_i^{e_i}$ then
\begin{equation}
  \Gal(\overline{\mathbb{F}_q}/\mathbb{F}_{q^S}) =  \prod_{p_i \text{prime}} p_i^{e_i} \mathbb{Z}_{p_i} 
\end{equation}
with~$p_i^{e_i}\mathbb{Z}_{p_i} = 0$ if~$e_i=\infty$. As an example, one can calculate~$\Gal(\overline{\mathbb{F}_q}/\mathbb{F}_{q^n})$ with~$n \in \mathbb{N}$. Since~$q^n$ is just another power of a prime~$p$ and we started with any power of~$p$, we should get~$\widehat{\mathbb{Z}}$ again. So if~$n = \prod_{p_i \text{prime}} p_i^{f_i}$ with all but finitely~$f_i$ equal to zero and all~$f_i < \infty$, we get
\begin{equation}
  \Gal(\overline{\mathbb{F}_q}/\mathbb{F}_{q^n})=\prod_{p_i \text{prime}} p_i^{f_i} \mathbb{Z}_{p_i}.
\end{equation}
But~$p_i^{f_i} \mathbb{Z}_{p_i} \cong \mathbb{Z}_{p_i}$ as a group (certainly not as a ring) if~$f_i < \infty$, so we get
\begin{equation}
  \prod_{p_i \text{prime}} p_i^{f_i} \mathbb{Z}_{p_i} \cong \prod_{p_i \text{prime}} \mathbb{Z}_{p_i} = \widehat{\mathbb{Z}}
\end{equation}
again. Calculating~$\Gal(\mathbb{F}_{q^n}/\mathbb{F}_{q})$, we get
\begin{equation}
  \widehat{\mathbb{Z}}/\prod_{p_i \text{prime}} p_i^{f_i} \mathbb{Z}_{p_i} = \prod_{p_i \text{prime}}\mathbb{Z}_{p_i}/\prod_{p_i \text{prime}} p_i^{f_i} \mathbb{Z}_{p_i} \cong \prod_{p_i \text{prime}} \mathbb{Z}/p_i^{f_i}\mathbb{Z},
\end{equation}
which by the Chinese Remainder Theorem is isomorphic to~$\mathbb{Z}/n\mathbb{Z}$, like we want.
