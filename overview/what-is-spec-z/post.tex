The mantra recited by~$\mathbb{F}_1$-followers is that~$\Spec(\mathbb{Z})$ is far too large to serve as the terminal object in the category of schemes, and, one should view it as a 'geometric' object over 'something' living 'under~$\mathbb{Z}$' called~$\mathbb{F}_1$: the field with one element.

In this seminar we will encounter a fair number of proposals as to what this elusive object~$\Spec(\mathbb{Z})$ viewed over~$\Spec(\mathbb{F}_1)$ might be. Let's start with the simplest and earliest proposal.

\emph{Smirnov's proposal} is that the smooth projective curve~$\Spec(\mathbb{Z})$ should have as its schematic points the set~$\{ 2,3,5,7,11,13,17,\cdots \} \cup \{ \infty \}$, that is, the set of all prime numbers together with~$\infty$, and, that the degree of the 'point'~$p$ should be equal to~$\log(p)$ whereas the degree of~$\infty$ is equal to~$1$.

\begin{figure}[ht]
  \centering
  \includegraphics[width=11cm]{what-is-spec-z/SpecZSmirnov}
  \caption{An interpretation of Smirnov's $\Spec\mathbb{Z}$}
  \label{figure:smirnov-spec-z}
\end{figure}

\subsection{Attempted explanation}
We have seen in Section~\ref{section:genus} that a schematic point~$P$ of a curve~$C$ defined over~$k$ corresponds to a discrete valuation ring in the function field~$k(C)$ and that its degree~$\deg(P)$ equals~$[\mathcal{O}_P/\mathfrak{m}_P : k]$.

By analogy, the schematic points of the 'projective curve'~$\Spec(\mathbb{Z})$ should correspond to all discrete valuations on~$\mathbb{Q}$, which by \href{http://en.wikipedia.org/wiki/Ostrowski's_theorem}{Ostrovski's theorem} are either the~$p$-adic valuations~$v_p(q)=n$ if~$q=p^n \frac{r}{s}$ and~$(r,p)=(s,p)=1$ or the real valuation~$v_{\infty}(q) = -\log |q|$ (minus sign because of the convention that the value of~$0$ should be~$\infty$).

To motivate the non-sensical definition of the degrees, recall that the degree of the divisor~$\div(f) = \sum_{P \in C} \ord_P(f) [P]$ equals zero for all~$f \in \overline{k}(C)$.

Now, if~$f$ is in the function field~$k(C)$, then its divisor must be invariant under the action of the Galois group~$\Gal(\overline{k}/k)$ (that is,~$\ord_{\sigma(P)}(f) = \ord_P(f)$ for all Galois-automorphisms~$\sigma$). But then, we can write~$\div(f)$ as a sum over the schematic points (which are the orbits of the geometric points under the action of the Galois group) and hence its degree is
\begin{equation}
  \deg(\div(P)) = \sum^{\text{scheme}}_{P \in C}\ord_P(f) \deg(P) = 0
\end{equation}
where now the sum is taken over all schematic points of~$C$. Once again, by analogy, if~$f = \pm \frac{p_1^{e_1} \cdots p_r^{e_r}}{q_1^{f_1} \cdots q_s^{f_s}} \in \mathbb{Q}$, then its 'divisor' is

$\div(f) = \sum_i e_i [p_i] - \sum_j f_j [q_j] - \log |f| [\infty]$

and Smirnov's proposal for the degrees of the scheme points of~$\Spec(\mathbb{Z})$ is (up to a common multiple) the only one assuring that the degree of all such divisors is zero.

\subsection{What is the field of constants?}

In this proposal~$\Spec(\mathbb{Z})$ is a smooth projective curve with function field~$\mathbb{Q}$. To determine the 'field' over which it is defined we have (in analogy with the functionfield case where the field of constants is~$K \cap \overline{k}$) to determine
\begin{equation}
  \mathbb{Q} \cap \overline{\mathbb{F}_1} = \mathbb{Q} \cap \pmb{\mu} = \{ +1,-1 \}
\end{equation}

So,~$\Spec(\mathbb{Z})$ is not really a curve over~$\mathbb{F}_1$, but rather over~$\mathbb{F}_{1^2}$.

\subsection{Smirnov's surface}

In~\cite{letters-to-manin} Smirnov explained that, as a first step towards the intersection theory on~$\Spec(\mathbb{Z}) \times \Spec(\mathbb{Z})$ (which might lead to a proof of the Riemann hypothesis by mimicking Weil's proof in the function field case) he embarked on the intersection theory in the somewhat easier product of two curves
\begin{equation}
  \mathbb{P}^1/\mathbb{F}_1\times\Spec(\mathbb{Z}).
\end{equation}
Combining the above with the description of the projective line over~$\mathbb{F}_1$ in Section~\ref{section:what-is-p1}, we can now depict this \emph{Smirnov surface} in Figure~\ref{figure:smirnov-spec-z-axes}.

Recall that the schematic points of~$\mathbb{P}^1~/~\mathbb{F}_1$ are~$\{ 0,\infty \} \cup \{  [1],[2],[3],\ldots \}$ where the point~$[n]$ represents all primitive~$n$-th roots of unity and so has degree~$\varphi(n)$.

\begin{figure}[ht]
  \centering
  \includegraphics[width=11cm]{what-is-spec-z/smirnovplane.jpg}
  \caption{$\mathbb{P}^1_{\mathbb{Z}}$ with its two axes}
  \label{figure:smirnov-spec-z-axes}
\end{figure}


