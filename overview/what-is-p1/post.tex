\label{section:what-is-p1}
\subsection{The short answer}
The geometric points of the projective line $\mathbb{P}^1$ over the 'field with one element' $\mathbb{F}_1$ form the set $\{ 0,\infty \} \cup \pmb{\mu}$ with $\pmb{\mu}$ the group of all roots of unity. Its schematic points form the set $\{ \infty,0 \} \cup \{ [1],[2],[3],[4],\cdots \}$ and the degree of the point $[n]$ equals $\phi(n)$.

\subsection{The longer answer}
We have seen in Section~\ref{section:genus} that the geometric points of $\mathbb{P}^1$ over the finite field $\mathbb{F}_p$ form the set

\begin{equation}
  \{ 0 = [0:1],~\infty=[1:0] \} \cup \{ \alpha = [\alpha:1]~:~\alpha \in \overline{\mathbb{F}}_p^{\ast} \}.
\end{equation}

I claim that the multiplicative group of the non-zero elements of the algebraic closure $\overline{\mathbb{F}}_p^{\ast}$ is isomorphic as group to the group $\pmb{\mu}^{(p)}$ of all roots of unity of order prime to $p$.

Clearly, any element $\alpha \in \mathbb{F}^{\ast}_{p^n}$   has order some divisor of $p^n-1$ and hence is prime to $p$. Conversely, if $(m,p)=1$ then $\overline{p}$ is a unit in $\mathbb{Z}/m\mathbb{Z}$ and therefore for some $n$ we have $p^n \cong 1~mod(m)$. But then, $m\divides p^n-1$ and there are primitive $m$-th roots of unity in $\mathbb{F}_{p^n}$.

However, describing this correspondence explicitly from a given construction of $\overline{\mathbb{F}}_p$ is very challenging. For example, \href{http://en.wikipedia.org/wiki/John_Horton_Conway}{John Conway} proved in \href{http://en.wikipedia.org/wiki/On_Numbers_and_Games}{ONAG}\todo{cite} that $\overline{\mathbb{F}}_2$ can be identified with all ordinals smaller than $\omega^{\omega^{\omega}}$ equipped with nim-addition and multiplication.

Finding the correspondence between small ordinals and odd roots of unity is the topic of the post \href{http://www.neverendingbooks.org/index.php/the-odd-knights-of-the-round-table.html}{The odd knights of the round table} (and follow-up posts \href{http://www.neverendingbooks.org/index.php/seating-the-first-few-thousand-knights.html}{here} and \href{http://www.neverendingbooks.org/index.php/seating-the-first-few-billion-knights.html}{here}).

<p>Below is the correpondence between $\mathbb{F}_{2^4}^{\ast}$ (identified with the ordinals from $1$ to $15$) and the 15-th roots of unity (nim-addition and nim-multiplication tables on the left). The lines describe the involution $x \mapsto x+1$.</p>

<p><img src="http://matrix.cmi.ua.ac.be/ngeometry/DATA/F16.jpg"></p>

The schematic points of $\mathbb{P}^1$ over $\mathbb{F}_p$ is the set of all $\Gal(\overline{\mathbb{F}}_p/\mathbb{F}_p) = \hat{\mathbb{Z}}$-orbits on the geometric points, and the degree of a scheme-point is the number of geometric points in the orbit.

Assigning to such a Galois-orbit $\mathcal{O}$ the polynomial $\prod_{\alpha \in \mathcal{O}} (x-\alpha)$ identifies the schematic points of $\mathbb{P}^1/\mathbb{F}_p$ with all irreducible polynomials in $\mathbb{F}_p[x]$ (together with $\infty$) and the point-degree coincides with the degree of the polynomial.

Concrete: say we have an explicit identification of $\mathbb{F}_{p^n}^{\ast}$ with all $p^n-1$-th roots of unity, then we can find all irreducible polynomials in $\mathbb{F}_p[x]$ of degree a divisor of $n$ by studying the orbits of these roots of unity under the power-map $z \mapsto z^p$.

In the picture above, we have indicated the different orbits of $\mathbb{F}_{2^4}$ with different colors. There are two orbits of length one : $\{ 0 \}$ corresponding to $x$ and $\{ 1 \}$ corresponding to $x+1$. One orbit of length two $\{ 2,3 \}$ corresponding to the irreducible polynomial $x^2+x+1$ (check the tables to verfify that this is indeed $(x-2)(x-3)$) and three orbits of length four

\begin{equation}
  \begin{aligned}
    \{ 4,6,5,7 \} &\leftrightarrow x^4+x+1 \\
    \{ 11,12,9,15 \} &\leftrightarrow x^4+x^3+1 \\
    \{ 14,8,13,10 \} &\leftrightarrow x^4+x^3+x^2+x+1
  \end{aligned}
\end{equation}

By \emph{analogy} we can now define the geometric points of $\mathbb{P}^1$ over the \href{http://en.wikipedia.org/wiki/Field_with_one_element}{field with one element} $\mathbb{F}_1$ to be the set $\{ 0,\infty \} \cup \pmb{\mu}^{(1)}$ where $\pmb{\mu}^{(1)}$ are all roots of unity of order prime to $1$, that is just all of them: $\pmb{\mu}$.

The schematic points of $\mathbb{P}^1/\mathbb{F}_1$ are then the orbits of this set under the action of the Galois group $\Gal(\mathbb{Q}(\pmb{\mu})/\mathbb{Q})$.

One checks that these orbits correspond to $\{ 0,\infty \}$ and $\{ [1],[2],[3],[4],\cdots \}$ where $[n]$ is the orbit consisting of all primitive $n$-th roots of unity. Consequently, the degree of the scheme-point $[n]$ is equal to $\phi(n)$ with $\phi$ the \href{http://en.wikipedia.org/wiki/Euler's_totient_function}{Euler function}.
