\label{advanced-alchemy}
Gauss dubbed it his ``golden theorem''. In this post we will replace~${\mathbb{Z}}$ by the structurally similar ring~${\mathbb{F}_{p}[X]}$ and prove higher versions of quadratic reciprocity, thereby turning gold into platinum.

\subsection{The golden theorem}

Remember that the classic quadratic reciprocity states that for~${p}$ and~${q}$ distinct and odd prime numbers, we have:
\begin{equation}
  \bigg( \frac{p}{q} \bigg) \bigg( \frac{q}{p} \bigg)=(-1)^{(\frac{p-1}{2})(\frac{q-1}{2})}= \left\{ \begin{array}{ll} 1 & \textrm{if~$p \equiv 1$ mod~$4$ or~$q \equiv 1$ mod~$4$}\\ -1 & \textrm{if~$p \equiv q \equiv 3$ mod~$4$} \end{array} \right.
\end{equation}
where~$\big( \frac{p}{q} \big)$ is the Legendre symbol, with value equal to~$1$ if the equation~$x^{2} \equiv p$ has a solution mod~$q$, and~$-1$ if it doesn't. To see why this is useful, consider the following question: find all odd primes~$p$ for which~$5$ is a quadratic residue mod~$p$. Using quadratic reciprocity, the solution is easy:
\begin{equation}
  \bigg( \frac{p}{5} \bigg) \bigg( \frac{5}{p} \bigg)=(-1)^{(\frac{p-1}{2})(\frac{5-1}{2})}=1,
\end{equation}
which holds iff~$(5/p)=(p/5)=1$ iff~$p=\pm 1\bmod5$. There are a number of other interesting applications, such as finding integer solutions to degree~$2$ polynomial equations.

\subsection{Legendre revisited}

Suppose you have an irreducible polynomial~${P}$ that does not divide a polynomial~${a}$, and~${d}$ a divisor of~${p-1}$, then by some elementary computations, the equation~${x^{d} \equiv a}$ has a solution mod~${P}$ iff~${a^{\frac{\vert P \vert -1}{d}} \equiv 1}$ mod~${P}$, with~${\vert P \vert = p^{deg(P)}}$. In general we will denote by~${(a/P)_{d}}$ the unique element in~${\mathbb{F}_{p}^{*} \subset (\mathbb{F}_{p}[X]/(P))^{*}}$, such that
\begin{equation}
  a^{\frac{\vert P \vert -1}{d}} \equiv \bigg( \frac{a}{P} \bigg)_{d}\bmod P.
\end{equation}
If~${P\,\vert\, a}$ we define~${(a/P)_{d}=0}$. This symbol is called the~${d}$-th power residue symbol.

\subsection{Carlitz' version}

Using this newly introduced symbol, we can state and prove the~${d}$-th power reciprocity law:
\begin{theorem}
  Let~${P}$ and~${Q}$ be monic, irreducible polynomials in~${\mathbb{F}_{p}[X]}$ of degrees~${\delta}$ and~${\nu}$ respectively. Then,
  \begin{equation}
    \bigg( \frac{Q}{P} \bigg)_{d} = (-1)^{\frac{p-1}{d}\delta \nu} \bigg( \frac{P}{Q} \bigg)_{d}.
  \end{equation}
  
  \begin{proof}
    The first step will consist of a simple reformulation of the statement. Let us define~${(a/P)=(a/P)_{q-1}}$. If we prove that
    \begin{equation}
      \bigg( \frac{Q}{P} \bigg) = (-1)^{\delta \nu} \bigg( \frac{P}{Q} \bigg),
    \end{equation}
    then the result will follow by raising both sides of the equality to the~${(p-1)/d}$ power and noting that~${(a/P)_{d}=(a/P)^{\frac{p-1}{d}}}$.
    
    Take~${\alpha}$ and~${\beta}$ roots of~${P}$ and~${Q}$ respectively. Over the field~${\mathbb{F}(\alpha,\beta)}$ we have
    \begin{equation}
      P(X)=(X-\alpha)(X - \alpha^{p}) \cdots (X-\alpha^{p^{\delta-1}})
    \end{equation}
    and
    \begin{equation}
      Q(X)=(X-\beta)(X-\beta^{p}) \cdots (X-\beta^{p^{\nu-1}}),
    \end{equation}
    using Fermat's little theorem and the Frobenius map.
    
    Noting that for a polynomial~${f}$ over~${\mathbb{F}(\alpha,\beta)}$, the equality~${f(X) \equiv f(\alpha)}$ holds mod~${(X-\alpha)}$, and for a polynomial~${g \in \mathbb{F}_{p}[X]}$ we have that~${g(X^{p})=g(X)^{p}}$, we can compute~${(Q/P)}$.
    \begin{align*} \bigg( \frac{Q}{P} \bigg) & \equiv Q(X)^{1+p+\cdots+p^{\delta-1}} \\ & \equiv Q(X)Q(X^{p}) \cdots Q(X^{p^{\delta -1}}) \\ & \equiv Q(\alpha)Q(\alpha^{p}) \cdots Q(\alpha^{p^{\delta-1}}) \bmod (X-\alpha) \end{align*}
    By invoking symmetry, this condition will also hold mod~${(X-\alpha^{p^{i}})}$ and thus also mod~${P}$. Plugging in the equation for~$Q(X)$, and noting that both sides of the resulting congruence,
    \begin{equation}
      \bigg( \frac{Q}{P} \bigg) \equiv \prod_{i=0}^{\delta -1} \prod_{j=0}^{\nu -1} (\alpha^{p^{i}} - \beta^{p^{j}}) \bmod P,  
    \end{equation}
    are in~${\mathbb{F}(\alpha,\beta)}$ (and thus equal), we get
    \begin{equation}
      \bigg( \frac{Q}{P} \bigg) = (-1)^{\delta \nu} \prod_{i=0}^{\delta -1} \prod_{j=0}^{\nu -1} (\beta^{p^{j}} - \alpha^{p^{i}}) = (-1)^{\delta \nu} \bigg( \frac{P}{Q} \bigg),  
    \end{equation}
    completing the proof.
  \end{proof}
\end{theorem}

