\section{Genus}
In these rough prep-notes, we are working towards the proof of the Riemann-Hurwitz genus formula\todo{refer to formula later on}. ``0-geometry'' means we want to use only fields and their discrete valuations so that we can port some of this later to number fields.

In the previous section we have seen that any field~$K$ of transcendence degree~$1$ over~$k$ with~$K \cap \overline{k} = k$ is really the function field~$K = k(C)$ of a smooth projective curve~$C$ defined over~$k$.

A geometric point~$P \in C$ is a discrete valuation ring~$\mathcal{O}_P$ in the extended field~$K^e = K \otimes \overline{k} = \overline{k}(C)$.

Our aim is to determine the genus of~$C$ from~$K^e$ and the discrete valuation rings~$\mathcal{O}_P$.

\subsection{Divisors}
For~$f \in K^e$ and~$P \in C$ we denote the valuation of~$f$ in the discrete valuation ring~$\mathcal{O}_P$ by~$\operatorname{ord}_P(f)$ (that is,~$f = u t^{\operatorname{ord}_P(f)}$ for~$t$ is a uniformizer and~$u$ a unit in~$\mathcal{O}_P$). 

We \emph{claim} that there are only finitely many~$P \in C$ such that~$\operatorname{ord}_P(f) \not= 0$ and that~$\sum_{P \in C} \operatorname{ord}_P(f) = 0$.

We can assume that~$f \notin \overline{k}$ and so the subring~$\overline{k}[f] \subset K^e$ is a polynomial ring. Let~$R$ be the integral closure of~$\overline{k}[f]$ in~$K^e$ (which is a finite field extension of~$\overline{k}(f)$ say of dimension~$r$). Then~$R$ is a Dedekind domain, projective of rank~$r$ over~$\overline{k}[f]$ and there are maximal ideals~$\mathcal{P}_i$ in~$R$ such that
\begin{equation}
  (f) = \mathcal{P}_1^{e_1} \cdots \mathcal{P}_s^{e_s}.
\end{equation}
Because the localization of~$R$ at~$\mathcal{P}_i$ is a discrete valuation ring with residue field~$\overline{k}$, each~$\mathcal{P}_i$ defines a point~$P_i \in C$ and we have~$\sum_i e_i = r$.

Similarly, let~$S$ be the integral closure of the polynomial algebra~$\overline{k}[\frac{1}{f}]$ in~$K^e$, then there are maximal ideals~$\mathcal{Q}_j$ (corresponding to points~$Q_j \in C$) such that
\begin{equation}
  \left(\frac{1}{f}\right) = \mathcal{Q}_1^{f_1} \cdots \mathcal{Q}_t^{f_t}
\end{equation}
and~$\sum_j f_j = r$. But then the \emph{divisor} of~$f$ satisfies the claims
\begin{equation}
  \operatorname{div}(f) = \sum_{P \in C} \operatorname{ord}_P(f) [P] = \sum_{i=1}^s e_i [P_i] - \sum_{j=1}^t f_j [Q_j].
\end{equation}


\subsection{Differential forms}
Consider the~$K^e$-vectorspace~$\Omega_C$ spanned by all 'differential forms'~$\mathrm{d}f$ where~$f \in K^e$, subject to the usual rules:
\begin{itemize}
  \item $\mathrm{d}(f+g)=\mathrm{d}f+\mathrm{d}g$ for all~$f,g \in K^e$;
  \item $\mathrm{d}(fg) = f\,\mathrm{d}g + g\,\mathrm{d}f$ for all~$f,g \in K^e$;
  \item $\mathrm{d}a = 0$ for all~$a \in \overline{k}$.
\end{itemize}

We \emph{claim} that~$\Omega_C$ has dimension one. More precisely, if~$x \in K^e$ is transcendental over~$\overline{k}$ such that~$K^e$ is a finite separable field extension of the subfield~$\overline{k}(x)$, then~$\Omega_C = K^e \mathrm{d}x$.

\begin{proof}
  The proof is a computation. Let~$g \in K^e$ have a minimal polynomial over~$\overline{k}(x)$ of the form
  \begin{equation}
    G(Y) = Y^n + f_1 Y^{n-1} + \cdots + f_{n-1} Y + f_n
  \end{equation}
  with all~$f_i \in \overline{k}(x)$. Now consider these two polynomials in~$\overline{k}(x)[Y]$: 
  \begin{equation}
    \begin{aligned}
      G_1(Y) &= n Y^{n-1} + (n-1) f_1 Y^{n-2} + \cdots + f_{n-1},\\
      G_2(Y) &= Y^n + \frac{\partial f_{1}}{\partial x} Y^{n-1} + \cdots + \frac{\partial f_{n-1}}{\partial x} Y + \frac{\partial f_n}{\partial x}
    \end{aligned}
  \end{equation}
  By the above equations among differential forms we get
  \begin{equation}
    0 = \mathrm{d} G(g) = G_2(g)\,\mathrm{d}x + G_1(g)\, \mathrm{d}g
  \end{equation}
  Because~$G_1(g) \not= 0$ by separability, it follows that~$\mathrm{d}g \in K^e \mathrm{d}x$. Done!
\end{proof}


\subsection{Genus}
In particular, if~$t$ is a uniformizing parameter of the discrete valuation ring~$\mathcal{O}_P$, then for any differential form~$\omega \in \Omega_C$ there is a unique~$f \in K^e$ such that~$\omega = f\,\mathrm{d}t$. We define~$\operatorname{ord}_P(\omega) =\operatorname{ord}_P(f)$. Clearly, this number depends only on~$\omega$ (and~$P$), but not on the choice of uniformizer (check!).

Slightly more involved is the \emph{claim} that~$\operatorname{ord}_P(\omega) \not= 0$ for  finitely many~$P \in C$. Here's the idea:

Take~$x \in K^e$ such that~$K^e$ is a finite separable extension of~$\overline{k}(x)$ of dimension~$r$, write~$\omega = f\,\mathrm{d}x$ and consider the corresponding cover~$x\colon C \rightarrow \mathbb{P}^1$. As before, there are at most~$r$ points of~$C$ lying over a point~$Q \in \mathbb{P}^1$.

Now, write~$K^e=\overline{k}(x)(\alpha)$ and let~$D \in \overline{k}(x)$ be the discriminant of the minimal polynomial of~$\alpha$ over~$\overline{k}(x)$. Then, away from the finite number of poles and zeroes of~$D$, there are precisely~$r$ points of~$C$ lying over any point~$Q \in \mathbb{P}^1$. 

So, removing a finite number of points from~$C$, in the remaining~$P \in C$ we have~$f(P) \not= 0,\infty$,~$x(P) \not= \infty$ and~$x-x(P)$ is a uniformizer of~$\mathcal{O}_P$. But in such points we have~$\operatorname{ord}_P(\omega) =\operatorname{ord}_P(f\,\mathrm{d}(x-x(P))) = 0$.

The number~$\sum_{P \in C} \operatorname{ord}_P(\omega)$ is thus well-defined and we \emph{claim} that it doesn't depend on the choice of differential form. For, any other form can be written as~$\omega' = f \omega$ for some~$f \in K^e$ and then we have
\begin{equation}
  \sum_{P \in C} \operatorname{ord}_P(\omega') = \sum_{P \in C} (\operatorname{ord}_P(f) +\operatorname{ord}_P(\omega))
\end{equation}
and we know already that~$\sum_{P \in C} \operatorname{ord}_P(f)=0$. The \emph{genus\emph}~$g_C$ of the curve~$C$ is then determined from that number by~$2g_C - 2 = \sum_{P \in C}\operatorname{ord}_P(\omega)$.

\begin{example}
  Take the projective line~$\mathbb{P}^1$ corresponding to the purely transcendental field~$\overline{k}(x)$ and consider~$\omega = \mathrm{d}x$. In a point~$\alpha \not= \infty$ we know that~$x-\alpha$ is a uniformizer, so
  \begin{equation}
    \operatorname{ord}_{\alpha}(\omega) = \operatorname{ord}_{\alpha}(\mathrm{d}x) = \operatorname{ord}_{\alpha}(\mathrm{d}(x-\alpha)) = 0.
  \end{equation}
  In~$\infty$ the uniformizer is~$\frac{1}{x}$, whence
  \begin{equation}
    \operatorname{ord}_{\infty}(\omega) =\operatorname{ord}_{\infty}(\mathrm{d}x) =\operatorname{ord}_{\infty}\left(-x^2\,\mathrm{d}\left(\frac{1}{x}\right)\right) = -2.
  \end{equation}
  Thus,~$\sum_{P \in \mathbb{P}^1} \operatorname{ord}_P(\omega) = -2$ and so the genus of the projective line~$g_{\mathbb{P}^1} = 0$.
\end{example}
