\section{On \texorpdfstring{$\mathsf{No}$}{No} and \texorpdfstring{$\mathsf{On}$}{On}}
Finite fields might seem to be easy to understand, but they are definitely not. Some examples of our limited knowledge are:
\begin{itemize}
  \item We represent the field~$\mathbb{F}_{p^n}$ by~$\mathbb{F}_p[x]/(f(x))$ whith~$f(x)$ a monic irreducible polynomial of degree~$n$. The polynomial~$f$ is not unique, and there is no best option.
  \item We can't extend~$\mathbb{F}_{p^n}$ to~$\mathbb{F}_{p^{n+1}}$ with compatible addition and multiplication.
  \item Simple problems like \href{http://www.neverendingbooks.org/index.php/the-odd-knights-of-the-round-table.html}{placing knights at a table} are too difficult.
  \item We know what the algebraic closure of a finite field is from Section~\ref{section:algebraic-closures-finite-fields}, but how can we do calculations in such an exotic thing?
\end{itemize}
\href{http://en.wikipedia.org/wiki/Conway_polynomial_\%28finite_fields\%29}{Conway polynomials} are a good example of the danger of solving these problems.

Therefore a better description of finite fields is necessary. For fields with characteristic two there is a good candidate using ordinal numbers with nim-addition and nim-multiplication. In this series I will work out this example. Let's start by defining numbers.

\subsection{Numbers}
In ONAG, Conway defines (surreal) numbers as follows:\todo{cite}

\begin{definition}
  If~$L$,~$R$ are any two sets of numbers, and no member of~$L$ is~$\geq$ any member of~$R$, then there is a number~$\{L \vert R\}$. All numbers are constructed in this way.
\end{definition}

The last sentence is somewhat informal. It means there is no sequence of numbers~$x_i = \{L_i \vert R_i\}$ with~$x_{i+1} \in L_i \cup R_i$ for all~$i \in \mathbb{N}$.

\begin{notation}
  If~$x = \{L \vert R\}$ we write~$x^L$ and~$x^R$ for the typical member of~$L$ resp.~$R$. For~$x$ itself we then write~$\{x^L \vert x^R\}$.
\end{notation}

Note that all the definitions are inductive, and don't need a basis because they start with the empty set. Each element of the empty set has all desired properties because the empty set has no members. This allows us to make a lot of proofs very short. If we can proof that for a property~$P$,~$P(x)$ is implied by~$P(x^L)$ and~$P(x^R)$, then~$P$ must be true for all numbers. Otherwise, there would be a number~$x$ for which~$P(x)$ is false. Hence, there is a~$x^L$ or~$x^R$ for which~$P(x^\bullet)$ is false. We can go on like this, but since that would create an infinite descending sequence of numbers,~$P(x)$ has to be true.

The relations on numbers are defined as:
\begin{itemize}
  \item~$x \leq y$ iff~$x < y^R$ and~$x^L < y$
  \item~$x \geq y$ iff~$x > y^L$ and~$x^R > y$
  \item~$x = y$ iff~$x \leq y$ and~$x \geq y$
\end{itemize}

This is a total order. A remarkable property is that~$x^L < x < x^R$ for all numbers~$x$. We now define an addition and multiplication. We know that~$x + y$ must lie between both~$x^L + y$ and~$x + y^L$ (on the left) and~$x^R + y$ and~$x + y^R$ (on the right). From~$x - x^L > 0$ and~$y - y^L > 0$ we can deduce~$(x - x^L)(y - y^L) > 0$, so that we must have~$xy > x^Ly + xy^L - x^Ly^L$. This motivates following definition:
\begin{equation}
  \mathrlap{
  \begin{aligned}
    x + y &\coloneqq \{x^L + y, x + y^L \vert x^R + y, x + y^R \} \\
    xy &\coloneqq \{ x^Ly + xy^L - x^Ly^L, x^Ry + xy^R - x^Ry^R \vert x^Ly + xy^R - x^Ly^R, x^Ry + xy^L - x^Ry^L \}
  \end{aligned}
  }
\end{equation}
With this addition and multiplication, the Class~$\textsf{No}$ of all surreal numbers forms a Field, but not a field (because~$\textsf{No}$ is not a set).

\subsection{Examples of numbers}
According to the definition, every number is constructed as two sets of \emph{earlier} constructed numbers. The only way to get it off the ground is by using empty sets. We call~$\{ \vert \}$ the number~$0$, born on the \emph{zeroth day}.

Starting from~$0$, the next generation of numbers can be constructed. We obtain~$1 = \{ 0 \vert \}$ and~$-1 = \{ \vert 0 \}$. It is easy to verify that~$\{ 0 \vert 0 \}$ is not a number. We have~$-1 < 0 < 1$. This was the \emph{first day}.

The number~$\{ 0 \vert 1 \}$ lies somewhere between~$0$ and~$1$. We call this number~$\frac{1}{2}$. Now what about~$\{ -1, 0 \vert 1 \}$? We can verify the inequalities~$\{ -1, 0 \vert 1 \} \leq \{ 0 \vert 1 \}$ and~$\{ -1, 0 \vert 1 \} \geq \{ 0 \vert 1 \}$, so we have two expressions for the same number. They are \emph{equal}, but not \emph{identical}.

We can go on like this with~$\frac{1}{4} = \{ 0 \vert \frac{1}{2} \}$ and~$\frac{3}{4} = \{ \frac{1}{2} \vert 1 \}$. The whole number~$n$ is born on day~$n$ as~$\{ n-1 \vert 0 \}$. All \href{http://en.wikipedia.org/wiki/Dyadic_rational}{dyadic rationals} and whole numbers are born on finite days. On day~$\omega$,~$L$ and~$R$ can be infinite sets. Examples are:
\begin{itemize}
  \item~$\frac{1}{3} = \{ \frac{1}{4}, \frac{1}{4} + \frac{1}{16}, \frac{1}{4} + \frac{1}{16} + \frac{1}{64}, \ldots \vert \frac{1}{2}, \frac{1}{2} - \frac{1}{8}, \ldots \}$
  \item~$\omega = \{ 0, 1, 2, 3, \ldots \vert \}$
  \item~$\frac{1}{\omega} = \{ 0 \vert 1, \frac{1}{2}, \frac{1}{4}, \frac{1}{8}, \ldots \}$
\end{itemize}
On day~$\omega + 1$ it's starting to get a little strange. The number~$x = \{ 0, 1, 2, 3, \ldots \vert \omega \}$ satisfies~$n < x < \omega$ for all finite integers~$n$, and~$x = \omega - 1$.

\subsection{Subsets}
The surreal numbers are the largest possible totally ordered field. The rationals, the reals, the ordinals and all other ordered fields are subsets of~$\textsf{No}$. An interesting subclass of~$\textsf{No}$ is~$\textsf{On}$, the class of all ordinal numbers.

\begin{definition}
 ~$\alpha$ is an \emph{ordinal number} if~$\alpha$ has an expression of the form~$\alpha = \{L\vert \}$.
\end{definition}

$\textsf{No}$ is a proper Class (not a set), but for every ordinal~$x$ the subclass~$\{ a : a < x \}$ is a set. Because~$\alpha = \{ \beta : \beta < \alpha \vert \}$, we can treat~$\alpha$ as the set of all lesser ordinals.
