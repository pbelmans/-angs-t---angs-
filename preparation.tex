\documentclass[11pt, a4paper, openany, oneside, article]{memoir}
\usepackage[fleqn, leqno]{amsmath}
\usepackage{amsthm}
\usepackage[english]{babel}
\usepackage{booktabs}
\usepackage{hyperref}
\usepackage{mathtools}
\usepackage{tikz}
\usepackage[normalem]{ulem}
\usepackage{todonotes}
\selectlanguage{english}

\usepackage[T1]{fontenc}
\usepackage[charter]{mathdesign}
\usepackage[scaled]{beramono, berasans}
\usepackage[draft = false]{microtype}
\frenchspacing

\counterwithout{section}{chapter}
\setcounter{secnumdepth}{2}
\setcounter{tocdepth}{2}

\newif\ifblog
\newif\iftex
\blogfalse
\textrue

\addtolength{\parskip}{.7ex}
\relpenalty=10000
\binoppenalty=10000 
\setlength\parindent{0cm}

\theoremstyle{definition}
\newtheorem{theorem}{Theorem}
\newtheorem{conjecture}[theorem]{Conjecture}
\newtheorem{consequence}[theorem]{Consequence}
\newtheorem{definition}[theorem]{Definition}
\newtheorem{example}[theorem]{Example}
\newtheorem{lemma}[theorem]{Lemma}
\newtheorem{remark}[theorem]{Remark}

\DeclareMathOperator\rad{rad}
\DeclareMathOperator\quality{q}

\begin{document}
\title{The $abc$-conjecture, the Mason-Stothers theorem and some generalities about both}
\author{Pieter Belmans}
\maketitle

\tableofcontents*

\clearpage


\section*{Introduction}

As my first attempt to student participation in the {\iftex\#\fi\ifblog#\fi}angs@t / angs+ seminar, I decided to tell something about the the~$abc$\nobreakdash-conjecture and the related Mason-Stothers theorem. The first has been hinted at but not yet formally introduced and the second is a statement for polynomials that is analogous to the first. And more importantly: it is a real theorem, with an easy proof. I will introduce both, give facts about both and will prove \sout{both} \emph{the latter}.

\iftex
The outline: in Section~\ref{section:abc} I will first give the formal statement of the $abc$-conjecture, it is already given online in \href{http://www.noncommutative.org/index.php/the-abc-conjecture.html}{the post of the same name} but we haven't seen it yet in the seminar. For personal reference I'll repeat this, together with some generalities about it. To end the section I discuss some consequences.
\fi
\ifblog
The outline: in this first post I will first give the formal statement of the $abc$-conjecture, it is already given online in \href{http://www.noncommutative.org/index.php/the-abc-conjecture.html}{the post of the same name} but we haven't seen it yet in the seminar. For personal reference I'll repeat this, together with some generalities about it. To end the section I discuss some consequences.
\fi

\iftex
Then in Section~\ref{section:mason-stothers} I will talk about the analogy for polynomials, resulting in the Mason-Stothers theorem. Afterwards I'll discuss the history of both the Mason-Stothers theorem and the~$abc$\nobreakdash-conjecture.
\fi
\ifblog
In the second post I will talk about the analogy for polynomials, resulting in the Mason-Stothers theorem. Afterwards I'll discuss the history of both the Mason-Stothers theorem and the~$abc$\nobreakdash-conjecture.
\fi

\iftex
To end my attempt I will give in Section~\ref{section:proof-and-consequences} an easy proof of the Mason-Stothers theorem and to conclude I will give some consequences, mainly the analogon of Fermat's last theorem for polynomials. The three big parts of this will be written as three separate posts, each is intended as material for a lecture of half an hour, somewhere in the seminar.
\fi
\ifblog
To end my attempt I will give in the third post an easy proof of the Mason-Stothers theorem and to conclude I will give some consequences, mainly the analogon of Fermat's last theorem for polynomials. The two post that follow this one will be put online when they are ready.
\fi

\clearpage

\iftex
\section[The abc-conjecture and some generalities]{The $abc$-conjecture and some generalities}
\label{section:abc}
\fi
\ifblog
\section{The $abc$-conjecture and some generalities}
\label{section:abc}
\fi

\iftex
\subsection[The abc-conjecture]{The $abc$-conjecture}
\fi
\ifblog
\subsection{The $abc$-conjecture}
\fi

Let's get \sout{physical} \emph{formal}.

\begin{conjecture}
  \label{conjecture:abc}
  If~$a$, $b$ and~$c$ are coprime positive integers such that
  \begin{equation}
    \label{equation:abc-equality}
    a+b=c
  \end{equation}
  then for~$\epsilon>0$ there are only finitely many triples~$(a,b,c)\in\mathbb{N}^3$ such that
  \begin{equation}
    \label{equation:abc-inequality}
    c>\rad(abc)^{1+\epsilon}.
  \end{equation}
\end{conjecture}

The function~$\rad\colon\mathbb{Z}\to\mathbb{N}$ maps an integer to the product of its prime factors. So we have~$\rad\colon n=\pm p_1^{e_1}\ldots p_k^{e_k}\mapsto p_1\ldots p_k$. What this conjecture now tries to convey is: in almost all cases we observe~$c<\rad(abc)$. I will give some examples.

\begin{example}
  \label{example:abc-1}
  Consider~$4+127=131$. We have~$\rad(4\cdot 127\cdot 131)=33274$ because~$127$ and~$131$ are both prime. And it is obvious that~$131<33274$.
\end{example}

\begin{example}
  \label{example:abc-2}
  Now for an example of a triple such that~$c>\rad(abc)$, consider the equality~$3+125=128$. We have~$\rad(3\cdot 125\cdot 128)=30$, which is considerably smaller than~$128$.
\end{example}

So why is the~$abc$\nobreakdash-conjecture this difficult? It's because we're relating additive structure (present in~\eqref{equation:abc-equality}) to multiplicative structure (present in~\eqref{equation:abc-inequality}). But maybe we could simplify the statement, dropping the~$+\epsilon$ in the exponent of the radical. We'd still be relating two different structures on~$\mathbb{Z}$ but at least it doesn't look like the~$\epsilon$-$\delta$-definition of continuity of real functions.

\begin{remark}
  The presence of an~$\epsilon$ in the exponent is necessary! If we drop it, reducing the exponential relationship between~$c$ and~$\rad(abc)$ to a \emph{linear} relationship, we'll get infinitely many~$(a,b,c)$ such that~$c>\rad(abc)$. I will now give a concrete example of an infinite class of counterexamples. For more general set-ups, please refer to~\cite{lower-bounds-abc-hits}.
  
  Take~$a=1$ and~$c=3^{2^k}$, so~$b$ is obviously~$3^{2^k}-1$. We need to determine how many factors of~$2$ there are in~$b$. Let's look at~$k=6$. We have
  \begin{equation}
    \begin{aligned}
      3^{64}-1&=(3^{32}+1)(3^{32}-1) \\
      &=\ldots \\
      &=(3^{32}+1)(3^{16}+1)\cdots(3+1)(3-1)
    \end{aligned}
  \end{equation}
  and every factor contributes a prime factor~$2$ but the second-to-last contributes two! So we end up with~$k+2$ times~$2$ in the general situation with~$b=3^{2^k}-1$. We obtain
  \begin{equation}
    \rad(b)\leq\frac{3^{2^k}-1}{2^{k+1}}<\frac{c}{2^{k}}
  \end{equation}
  because we can isolate the factor~$2$ from the radical knowing about its presence and dividing the remaining factors from~$b$. This leads to
  \begin{equation}
    \rad(abc)=3\rad(b)<\frac{3c}{2^k}\quad\Leftrightarrow\quad c>\rad(abc)\frac{2^k}{3}
  \end{equation}
  hence for~$k\geq 2$ we have a triple such that the linear variant of the inequality is satisfied. We definitely need the presence of an~$\epsilon$ of extra space in the exponent.
\end{remark}

\subsection{Another viewpoint}

For some more facts, I'll introduce the notion of quality, which is a measurement of the relation between~$c$ and~$\rad(abc)$.

\begin{definition}
  For integers~$a$, $b$ and~$c$ define the \emph{quality} of the triple~$(a,b,c)$ as
  \begin{equation}
    \quality(a,b,c)\coloneqq\frac{\ln(c)}{\ln\left( \rad(abc) \right)}.
  \end{equation}
\end{definition}

Now the~$abc$\nobreakdash-conjecture has a nice equivalent statement using this notion of quality:

\begin{conjecture}
  There are only finitely many~$(a,b,c)\in\mathbb{N}^3$ subject to the conditions in Conjecture~\ref{conjecture:abc} such that for~$\epsilon>0$ we have~$\quality(a,b,c)>1+\epsilon$.
\end{conjecture}

If we translate Examples~\ref{example:abc-1} and~\ref{example:abc-2} to this new concept we end up with
\begin{equation}
  \begin{aligned}
    \quality(4,127,131)&=0.468\ldots \\
    \quality(3,125,128)&=1.427\ldots
  \end{aligned}
\end{equation}

Now you could ask yourself the question: how big can~$q$ get for any triple~$(a,b,c)$? Well, the theoretical answer is obviously beyond reach, just like the actual proof of the~$abc$\nobreakdash-con\-jecture, but  the distributed comping project ABC@Home is trying to get as much information from examples as possible, enumerating all triples~$(a,b,c)$ such that~$c<10^{20}$ and extracting information out of them. As of now, they are currently somewhere around~$c\approx10^{18}$ but refer to~\iftex\href{http://abcathome.com/data/}{\texttt{abcathome.com/data}}\fi\ifblog\href{http://abcathome.com/data/}{abcathome.com/data}\fi for more information.

We get the following table of highscores:

\begin{center}
  \begin{tabular}{cllll}
     & $\quality(a,b,c)$ & $a$ & $b$ & $c$ \\\iftex\midrule\fi
    $1$ & $1.6299\ldots$ & $2$ & $3^{10}109$ & $23^5$ \\
    $2$ & $1.6260\ldots$ & $11^2$ & $3^25^67^3$ & $2^{21}23$ \\
    $3$ & $1.6235\ldots$ & $19\cdot1307$ & $7\cdot29^231^8$ & $2^83^{22}5^4$ \\
    $4$ & $1.5808\ldots$ & $283$ & $5^{11}13^2$ & $2^83^817^3$ \\
    $5$ & $1.5679\ldots$ & $1$ & $2\cdot3^7$ & $5^47$ \\
  \end{tabular}
\end{center}

which summarizes~\iftex\href{http://www.math.leidenuniv.nl/~desmit/abc/index.php?set=2}{\texttt{math.leidenuniv.nl/~desmit/abc/index.php?set=2}}\fi\ifblog\href{http://www.math.leidenuniv.nl/~desmit/abc/index.php?set=2}{math.leidenuniv.nl/~desmit/abc/index.php?set=2}\fi. Almost one year ago, only 234 triples with~$q>1.4$ were known.

But this computational approach is only interesting for information gathering, as Gauss and Riemann (not quite the least mathematicians) once conjectured while working on the prime number theorem that~$\pi(x)$ (the number of primes below~$x$) is strictly smaller than the logarithmic integral
\begin{equation}
  \operatorname{li}(x)=\int_0^x\frac{\mathrm{d}t}{\ln t}
\end{equation}
but it is proved that there \emph{must} be a counterexample smaller than~$1.4\cdot10^{316}$ but up to~$10^{14}$ none have been found so far. And there are infinitely many counterexamples too.

\subsection{Consequences}

Wikipedia gives a list of consequences ranging from absolutely stupendous important to nice facts about prime numbers. A short selection, which has been assembled based on my personal taste and understanding of the statements:

\begin{description}
  \item [\textbf{Fermat's last theorem}] Proved by Wiles in 1995 and with the overly known statement: there are no non-trivial integers such that~$x^n+y^n=z^n$ for~$n\geq 3$. An easy consequence of a proof of the~$abc$\nobreakdash-conjecture would be Fermat-Wiles for ``sufficiently large exponents''. In case you wonder what sufficiently large means: we don't need an astronomical number like the bound from counterexamples for~$\pi(x)>\operatorname{li}(x)$, just~$n\geq 6$ will do (under a mild assumption). I will prove this below.

  \item[\textbf{Faltings' theorem}] Curves of genus greater than~$1$ over~$\mathbb{Q}$ have only finitely many rational points. This used to be Mordell's conjecture before, where Mordell is the one of Mordell-Weil fame (the abelian group of rational points is finitely generated).

  \item[\textbf{Infinitely many non-Wieferich primes}] A Wieferich prime is a prime number such that~$p^2\,|\, 2^{p-1}-1$. At this moment only two Wieferich primes are known: 1093 and 3511. There are three scenarios possible: both sets are infinite or one of them is finite. But a proof of the $abc$\nobreakdash-conjecture would only rule out the scenario where there are finitely many non-Wieferich primes, which by numerical evidence is quite unlikely.

  \item[\textbf{Fermat-Catalan conjecture}] This is a combination of Fermat's last theorem as discussed above and the Catalan conjecture. As this conjecture was proved in 2002 by Mih\u{a}ilescu this name is quite ambiguous, but apparently calling it Mih\u{a}ilescu's theorem isn't common (which is obvious after writing his name twice). This theorem states that~$2^3$ and~$3^2$ are the only powers of natural numbers that are consecutive. More formally:
    \begin{equation}
      x^a-y^b=1
    \end{equation}
    where~$a,b,x,y>1$ only admit the described solution.

    Combining Fermat and Catalan we obtain that
    \begin{equation}
      a^m+b^n=c^k
    \end{equation}
    has only finitely many solutions where~$a,b,c$ are coprime natural numbers and~$m,n,k$ are positive integers such that
    \begin{equation}
      \frac{1}{m}+\frac{1}{n}+\frac{1}{k}<1.
    \end{equation}

  \item[\textbf{Szpiro's conjecture}] Unlike the previous ones, this is \emph{equivalent} to the statement of the~$abc$-conjecture. Due to its technicalities (elliptic curve stuff) I won't state it here, but it's nice to know there are equivalent statements in a different language.
\end{description}

As you can see: there are both theorems and conjectures. Some conjectures have overwhelming numerical evidence and 


\begin{proof}[Proof of $abc$-conjecture implies Fermat for~$n\geq 6$]
  One can reduce the statement of Fermat's last theorem (but you should look at it as a conjecture now, whilst the~$abc$\nobreakdash-conjecture has just been proven by yourself and you wish to earn hard cash proving Fermat's conjecture) to a scenario with~$x$, $y$ and~$z$ coprime. Now let~$a=x^n$, $b=y^n$ and~$c=z^n$. We have
  \begin{equation}
    \rad(x^ny^nz^n)\leq xyz\leq z^3
  \end{equation}

  Now assume that we have~$\epsilon_0>0$ such that there are \emph{no} solutions to the scenario of the~$abc$\nobreakdash-theorem, which must be possible as any~$\epsilon>0$ admits only finitely many counterexamples. We now obtain
  \begin{equation}
    z^n<(z^3)^{\epsilon_0}
  \end{equation}
  so Fermat's conjecture has now been proved for~$n\geq 3\epsilon_0$!

  It is believed that~$\epsilon_0\leq 2$ by the way.
\end{proof}

\clearpage


\section{The Mason-Stothers theorem and a history of both}
\label{section:mason-stothers}

\subsection{The Mason-Stothers theorem}

Enough conjectural stuff, time for some \emph{facts}. Did you know~$\mathbb{Z}$ is an Euclidean domain?

\begin{definition}
  An \emph{Euclidean domain} is an integral domain~$R$ equipped with an \emph{Euclidean function}~$d\colon R\setminus\left\{ 0 \right\}\to\mathbb{N}$ such that for~$a$ and~$b$ in~$R$ with~$b\neq 0$ we have a division with a remainder, i.e., there exist~$q$ and~$r$ in~$R$ such that
  \begin{equation}
    a=bq+r
  \end{equation}
  and either~$r=0$ or~$d(r)<d(b)$. An additional requirement could be~$d(a)\leq d(ab)$ for nonzero~$a$ and~$b$ but it can be proved that every Euclidean function can be modified in order to satisfy this extra property.
\end{definition}

\begin{remark}
  An Euclidean domain can be equipped with \emph{several} Euclidean functions. When there exists a single function we can call~$R$ Euclidean, but all scalar multiples suffice too for instance.
\end{remark}

But~$\mathbb{Z}$ is not the only Euclidean domain. The polynomials~$k[X]$ over a field~$k$ can be used too! Now take~$k$ to be an algebraically closed field of characteristic~$0$. This will be necessary: we'll split polynomials into linear factors and we'll use the first derivative. These will break in non-algebraically closed fields or fields with characteristic different from~$0$. What happens in~$\mathbb{F}_1$ is left as a trivial exercise.

\begin{theorem}[Mason-Stothers]
  \label{theorem:mason-stothers}
  Let~$a(X)$, $b(X)$ and~$c(X)$ be three coprime polynomials (i.e., not sharing a common factor) such that
  \begin{equation}
    \label{equation:mason-stothers-equality}
    a(X)+b(X)=c(X)
  \end{equation}
  then there are \emph{no} triples~$(a(X),b(X),c(X))\in k[X]^3$ such that
  \begin{equation}
    \max\left\{ \deg a(X),\deg b(X),\deg c(X) \right\}>\deg\rad(abc(X))-1.
  \end{equation}
\end{theorem}

The function~$\rad\colon k[X]\to k[X]$ now maps a polynomial~$f(X)$ to the polynomial of minimal degree such that it has the same roots as~$f$. Under the assumption of an algebraically closed field this is the same as removing the multiplicities of the linear terms. Can you see the analogy?

\begin{remark}
  My statement of the~$abc$\nobreakdash-conjecture was phrased in terms non-negative integers, removing the need for a maximum over the values of the Euclidean function~$|\cdot|$. The post introducing the~$abc$\nobreakdash-conjecture states the inequality in this more general form. In the statement of the Mason-Stothers theorem the presence of an Euclidean function is more obvious.
\end{remark}

But have you noticed it? The statement doesn't require an~$\epsilon$ and there are not finitely many counterexamples: there simply are \emph{no} counterexamples to the inequality! See Section~\ref{section:proof-and-consequences} for an actual proof. But first I want to throw some dates at you.

\subsection{History}

\begin{description}
  \item[\textbf{1981}] Wilson Stothers states the theorem in his paper Polynomial identities and Hauptmoduln but the theorem is not remarked upon.
  \item[\textbf{1983}] R.\ Mason rediscovers the theorem and publishes it in his book Diophantine Equations over Function Fields.
  \item[\textbf{1985}] Joseph Oesterl\'e and David Masser conjecture what is now known as the~$abc$\nobreakdash-conjecture but might as well be (and sometimes is) called the Oesterl\'e-Masser conjecture.
  \item[\textbf{2000}] Noah Snyder, while still in high school, gives an elementary proof of the Mason-Stothers theorem in the paper An alternate proof of Mason's theorem, which I will reproduce in Section~\ref{section:proof-and-consequences}.
  \item[\textbf{2006}] ABC@Home begins its search.
\end{description}

\clearpage


\section{Proof of the Mason-Stothers theorem and some consequences}
\label{section:proof-and-consequences}

\subsection{Proof}

Consider the statement of Theorem~\ref{theorem:mason-stothers}. We'll prove some lemmas first. We'll denote the derivative of a polynomial~$f$ by $f'$.

\begin{lemma}
  Let~$f(X)\in k[X]$ then~$f$ has a multiple root if and only if~$f$ and~$f'$ have a common root.

  \begin{proof}
    If~$f$ has a root~$\alpha$ of multiplicity~$n$ it can be written as~$(X-\alpha)^ng(X)$ where~$g(\alpha)\neq 0$. Now~$f'$ is given by
    \begin{equation}
      n(X-\alpha)^{n-1}g(X)+(X-\alpha)^ng'(x)=(X-\alpha)^{n-1}\left( ng(X)+(x-\alpha)g(X) \right).
    \end{equation}
    If~$n=1$ evaluating~$f'$ in~$\alpha$ yields~$ng(\alpha)\neq 0$ as we've assumed characteristic~$0$. Likewise~$n\geq 2$ yields~$f'(\alpha)=0$ because~$(X-\alpha)^{n-1}$ doesn't vanish.
  \end{proof}
\end{lemma}

\begin{lemma}
  Let~$f(X)\in k[X]$ then~$\deg\gcd(f,f')=\deg f-\deg\rad(f)$.

  \begin{proof}
    Because we've assumed that~$k$ is algebraically closed we can write~$f$ as
    \begin{equation}
      \prod_{i=1}^m(X-\alpha_i)^{n_i}
    \end{equation}
    and~$\sum_{i=1}^mn_i=\deg f$. We now obtain
    \begin{equation}
      \gcd(f,f')=\prod_{i=1}^m(X-\alpha_i)^{n_i-1}
    \end{equation}
    so
    \begin{equation}
      \begin{aligned}
        \deg\gcd(f,f')&=\sum_{i=1}^m(n_i-1) \\
        &=\left( \sum_{i=1}^mn_i \right)-m \\
        &=\deg f-m
        &=\deg f-\deg\rad(f).
      \end{aligned}
    \end{equation}
  \end{proof}
\end{lemma}

We are now ready for the actual proof. No need to develop an intersection theory on~$\mathbb{P}^1/\mathbb{F}_1\times\operatorname{Spec}\mathbb{Z}$.

\begin{proof}[Proof of the Mason-Stothers theorem]
  We have
  \begin{equation}
    \begin{aligned}
      a'(X)b(X)-a(X)b'(X)&=a'(X)\left( c(X)-a(X) \right)-a(X)\left( c'(X)-a'(X) \right) \\
      &=a'(X)c(X)-a(X)c'(X)
    \end{aligned}
  \end{equation}
  by the linearity of the derivative.

  Now at least two of our chosen polynomials must be non-constant, otherwise the third by~\eqref{equation:mason-stothers-equality} is constant too. Assume~$a(X)$ and~$b(X)$ are non-constant, we obtain~$a'(X)b(X)-a(X)b'(X)\neq 0$ \dots
\end{proof}

\subsection{Consequences}


\bibliographystyle{alpha}
\bibliography{bibliography}

\end{document}
