\documentclass[11pt, a4paper, openany, oneside, article]{memoir}
\usepackage[fleqn, leqno]{amsmath}
\usepackage{amsthm}
\usepackage[english]{babel}
\usepackage{booktabs}
\usepackage{hyperref}
\usepackage{mathtools}
\usepackage{tikz}
\usepackage[normalem]{ulem}
\selectlanguage{english}

\usepackage[T1]{fontenc}
\usepackage[charter]{mathdesign}
\usepackage[scaled]{beramono, berasans}
\usepackage[draft = false]{microtype}
\frenchspacing

\counterwithout{section}{chapter}

\usetikzlibrary{matrix}

\addtolength{\parskip}{.7ex}
\relpenalty=10000
\binoppenalty=10000 
\setlength\parindent{0cm}

\theoremstyle{definition}
\newtheorem{theorem}{Theorem}
\newtheorem{conjecture}[theorem]{Conjecture}
\newtheorem{consequence}[theorem]{Consequence}
\newtheorem{definition}[theorem]{Definition}
\newtheorem{example}[theorem]{Example}
\newtheorem{remark}[theorem]{Remark}

\DeclareMathOperator\rad{rad}
\DeclareMathOperator\quality{q}

\begin{document}
\title{\textls[20]{The Mason-Stothers theorem and some facts about the~$abc$-conjecture}}
\author{Pieter Belmans}
\maketitle

\section{Introduction}
\label{section:introduction}

As the first attempt to student participation in the \#angs@t / angs+ seminar, I decided to tell something about the Mason-Stothers theorem. This is a statement for polynomials that is analogous to the major theme of this seminar: the~$abc$\nobreakdash-conjecture. And more importantly: it is a real theorem, with an easy proof. Alongside I will give some facts annex trivialities about the~$abc$\nobreakdash-conjecture.

The outline: in Section~\ref{section:introduction} I will first give the formal statement of the $abc$-conjecture, it is already given online in \href{http://www.noncommutative.org/index.php/the-abc-conjecture.html}{the post of the same name} but we haven't seen it yet in the seminar. For personal reference I'll repeat this, together with the promised trivialities. Then I will talk about the analogy, resulting in the Mason-Stothers theorem. To end the introduction I will discuss the history of both the Mason-Stothers theorem and the~$abc$\nobreakdash-conjecture. In Section~\ref{section:statement-and-proof} I will give the theorem and the proof. To conclude I will give some consequences, mainly the analogon of Fermat's last theorem for polynomials.

\subsection{The $abc$-conjecture}

Let's get \sout{physical} \emph{formal}.

\begin{conjecture}
  \label{conjecture:abc}
  If~$a$, $b$ and~$c$ are coprime positive integers such that
  \begin{equation}
    \label{equation:abc-equality}
    a+b=c
  \end{equation}
  then for~$\epsilon>0$ there are only finitely many~$(a,b,c)\in\mathbb{N}^3$ such that
  \begin{equation}
    \label{equation:abc-inequality}
    c>\rad(abc)^{1+\epsilon}.
  \end{equation}
\end{conjecture}

The function~$\rad\colon\mathbb{N}\to\mathbb{N}$ maps an integer to the product of its prime factors. So we have~$\rad\colon n=p_1^{e_1}\ldots p_k^{e_k}\mapsto p_1\ldots p_k$. What this conjecture now tries to convey is: in almost all cases we observe~$c<\rad(abc)$. I will give some examples.

\begin{example}
  \label{example:abc-1}
  Consider~$4+127=131$. We have~$\rad(4\cdot 127\cdot 131)=33274$ because~$127$ and~$131$ are both prime. And it is obvious that~$131<33274$.
\end{example}

\begin{example}
  \label{example:abc-2}
  Now for an example of a triple such that~$c>\rad(abc)$, consider the equality~$3+125=128$. We have~$\rad(3\cdot 125\cdot 128)=30$, which is considerably smaller than~$128$.
\end{example}

So why is the~$abc$\nobreakdash-conjecture this difficult? It's because we're relating additive structure (present in~\eqref{equation:abc-equality}) to multiplicative structure (present in~\eqref{equation:abc-inequality}). But maybe we could simplify the statement, dropping the~$+\epsilon$ in the exponent of the radical. We'd still be relating two different structures on~$\mathbb{Z}$ but at least it doesn't look like the~$\epsilon$-$\delta$-definition of continuity of real functions.

\begin{remark}
  The presence of an~$\epsilon$ in the exponent is necessary! If we drop it, reducing the exponential relationship between~$c$ and~$\rad(abc)$ to a \emph{linear} relationship, we'll get infinitely many~$(a,b,c)$ such that~$c>\rad(abc)$. I will now give a concrete example of an infinite class of counterexamples. For more general set-ups, please refer to~\cite{lower-bounds-abc-hits}.
  
  Take~$a=1$ and~$c=3^{2^k}$, so~$b$ is obviously~$3^{2^k}-1$. We need to determine how many factors of~$2$ there are in~$b$. Let's look at~$k=6$. We have
  \begin{equation}
    \begin{aligned}
      3^{64}-1&=(3^{32}+1)(3^{32}-1) \\
      &=\ldots \\
      &=(3^{32}+1)(3^{16}+1)\cdots(3+1)(3-1)
    \end{aligned}
  \end{equation}
  and every factor contributes a prime factor~$2$ but the second-to-last contributes two! So we end up with~$k+2$ times~$2$ in the general situation with~$b=3^{2^k}-1$. We obtain
  \begin{equation}
    \rad(b)\leq\frac{3^{2^k}-1}{2^{k+1}}<\frac{c}{2^{k}}
  \end{equation}
  because we can isolate the factor~$2$ from the radical knowing about its presence and dividing the remaining factors from~$b$. This leads to
  \begin{equation}
    \rad(abc)=3\rad(b)<\frac{3c}{2^k}\quad\Leftrightarrow\quad c>\rad(abc)\frac{2^k}{3}
  \end{equation}
  hence for~$k\geq 2$ we have a triple such that the linear variant of the inequality is satisfied. We definitely need the presence of an~$\epsilon$.
\end{remark}

For some more facts, I'll introduce the notion of quality, which is a measurement of the relation between~$c$ and~$\rad(abc)$.

\begin{definition}
  For integers~$a$, $b$ and~$c$ define the \emph{quality} of the triple~$(a,b,c)$ as
  \begin{equation}
    \quality(a,b,c)\coloneqq\frac{\ln(c)}{\ln\left( \rad(abc) \right)}.
  \end{equation}
\end{definition}

Now the~$abc$\nobreakdash-conjecture has a nice equivalent statement using this notion of quality:

\begin{conjecture}
  There are only finitely many~$(a,b,c)\in\mathbb{N}^3$ subject to the conditions in Conjecture~\ref{conjecture:abc} such that for~$\epsilon>0$ we have~$\quality(a,b,c)>1+\epsilon$.
\end{conjecture}

If we translate Examples~\ref{example:abc-1} and~\ref{example:abc-2} to this new concept we end up with
\begin{equation}
  \begin{aligned}
    \quality(4,127,131)&=0.468\ldots \\
    \quality(3,125,128)&=1.427\ldots
  \end{aligned}
\end{equation}

Now you could ask yourself the question: how big can~$q$ get for any triple~$(a,b,c)$? Well, the theoretical answer is obviously beyond reach, just like the actual proof of the~$abc$\nobreakdash-con\-jecture, but  the distributed comping project ABC@Home is trying to get as much information from examples as possible, enumerating all triples~$(a,b,c)$ such that~$c<10^{20}$ and extracting information out of them. We get the following table of highscores:

\begin{center}
  \begin{tabular}{cllll}
     & $\quality(a,b,c)$ & $a$ & $b$ & $c$ \\\midrule
    $1$ & $1.6299\ldots$ & $2$ & $3^{10}109$ & $23^5$ \\
    $2$ & $1.6260\ldots$ & $11^2$ & $3^25^67^3$ & $2^{21}23$ \\
    $3$ & $1.6235\ldots$ & $19\cdot1307$ & $7\cdot29^231^8$ & $2^83^{22}5^4$ \\
    $4$ & $1.5808\ldots$ & $283$ & $5^{11}13^2$ & $2^83^817^3$ \\
    $5$ & $1.5679\ldots$ & $1$ & $2\cdot3^7$ & $5^47$ \\
  \end{tabular}
\end{center}

which summarizes~\href{http://www.math.leidenuniv.nl/~desmit/abc/index.php?set=2}{\texttt{math.leidenuniv.nl/~desmit/abc/index.php?set=2}}. Almost one year ago, only 234 triples with~$q>1.4$ were known.

\begin{enumerate}
  \item Introducing the $abc$-conjecture
  \item Remarking that polynomials have a analogous properties
  \item History
\end{enumerate}


\section{Formal statement and proof}
\label{section:statement-and-proof}

\begin{enumerate}
  \item Formal statement of the theorem
  \item Proof
  \item Consequences
\end{enumerate}


\bibliographystyle{alpha}
\bibliography{bibliography}

\end{document}
